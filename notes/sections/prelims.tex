\section{Preliminaries}

\subsection{Assumptions}

Let $\mathcal{G}$ be an algorithm that on input $1^\lambda$ uniformly samples from a sequence of group distributions $(S_\lambda)_{\lambda\in\Z\geq 0}$ of group distributions of the form $\Gamma[\hat{\mathbb{G}}, \mathbb{G}, g, q]\in[S_\lambda]$, where $\hat{\mathbb{G}}$ is a finite multiplicative abelian group, $\G$ is a prime-order subgroup, $g$ is a generator of $\G$ and $q$ is the order of $\G$. We assume that group operations in $\G$ can be done in (expected) polynomial time in $\G$, including exponentiation, selecting a random group element, and membership. 

Note that a random group element can be chosen by selecting $x\rgets\Z_q$ and computing $g^x$. We define $\bar\G$ to be $\G\setminus\{1\}$; since $q$ is prime, $\bar\G$ is the set of generators of $\G$. If $x\rgets\bar\G$, then we obtain a random generator.

\subsubsection{The CDH Assumption}

Let $\mathcal{G}$ be as above. For all probabilistic polynomial-time $\adv$ and for each $\lambda\in\Z_{\geq 0}$, we define the CDH advantage of $\adv$ against $\mathcal{G}$ at $\lambda$ given $\Gamma$ as

$$\advcdh_{\mathcal{G},\adv}(\lambda | \Gamma):\Pr[c=g^{xy}: x, y\rgets \Z_q, c\rgets\adv(1^\secpar,\Gamma,g^x,g^y)].$$

The Computational Diffie-Hellman (CDH) assumption for $\mathcal{G}$ states that for every probabilistic, polynomial-time algorithm $\adv$, there exists some negligible function $\negl$ such that for all $\lambda\in\Z_{\geq 0}$, $$\advcdh_{\mathcal{G},\adv}(\lambda | \Gamma)\leq\negl(\lambda)$$ where $\Gamma\rgets[S_\lambda]$.

\subsubsection{The DDH Assumption}

Let $\mathcal{G}$ be as above. We define for each $\lambda\in\Z_{\geq 0}$ and for all $\Gamma[\hat\G, \G, g, q]\in[S_\lambda]$ the sets 
\begin{equation*}
	\randdh_{\lambda,\Gamma} := \{(g, g^x, g^y, g^{z}) : x,y\rgets\Z_q^{*}, z\rgets\Z_q\}
\end{equation*}
and
\begin{equation*}
	\ddh_{\lambda,\Gamma} := \{(g, g^x, g^y, g^{xy}) : x,y\rgets\Z_q^{*}\}.
\end{equation*}
Where $\ddh_{\lambda,\Gamma}$ is the set of Diffie-Hellman Triples. For any probabilistic polynomial-time distinguishing algorithm $\adv$ and for all $\lambda\in\Z_{\geq 0}$, we define the DDH advantage of $\adv$ against $\mathcal{G}$ at $\lambda$ given $\Gamma$ as
\begin{equation*}
	\advddh_{\mathcal{G},\adv}(\lambda | \Gamma) = \left|\Pr[\adv(1^\lambda,\Gamma,\rho) = 1 : \rho\rgets\randdh_{\lambda,\Gamma}] - \Pr[\adv(1^\lambda,\Gamma,\rho) = 1 : \rho\rgets\ddh_{\lambda,\Gamma}]\right|
\end{equation*}
The Decisional Diffie-Hellman (DDH) assumption for $\mathcal{G}$ states that for every probabilistic, polynomial-time algorithm $\adv$, there exists some negligible function $\negl$ such that for all $\lambda\in\Z_{\geq 0}$, $$\advddh_{\mathcal{G},\adv}(\lambda | \Gamma)\leq\negl(\lambda)$$ where $\Gamma\rgets[S_\lambda]$.

\subsubsection{The DSDH Assumption}

Let $\mathcal{G}$ be as above. We define for each $\lambda\in\Z_{\geq 0}$ and for all $\Gamma[\hat\G, \G, g, q]\in[S_\lambda]$ the sets 
\begin{equation*}
	\randsdh_{\lambda,\Gamma} := \{(g, g^x, g^{y}) : x,y\rgets\Z_q^{*}\}
\end{equation*}
and
\begin{equation*}
	\sdh_{\lambda,\Gamma} := \{(g, g^x, g^{x^2}) : x\rgets\Z_q^{*}\}.
\end{equation*}
Where $\sdh_{\lambda,\Gamma}$ is the set of Square Diffie-Hellman Triples. For any probabilistic polynomial-time distinguishing algorithm $\adv$ and for all $\lambda\in\Z_{\geq 0}$, we define the DSDH advantage of $\adv$ against $\mathcal{G}$ at $\lambda$ given $\Gamma$ as
\begin{equation*}
	\advsdh_{\mathcal{G},\adv}(\lambda | \Gamma) = \left|\Pr[\adv(1^\lambda,\Gamma,\rho) = 1 : \rho\rgets\randsdh_{\lambda,\Gamma}] - \Pr[\adv(1^\lambda,\Gamma,\rho) = 1 : \rho\rgets\sdh_{\lambda,\Gamma}]\right|
\end{equation*}
The Decisional Square Diffie-Hellman (DSDH) assumption for $\mathcal{G}$ states that for every probabilistic, polynomial-time algorithm $\adv$, there exists some negligible function $\negl$ such that for all $\lambda\in\Z_{\geq 0}$, $$\advsdh_{\mathcal{G},\adv}(\lambda | \Gamma)\leq\negl(\lambda)$$ where $\Gamma\rgets[S_\lambda]$.

\subsubsection{One-Time Signature Schemes} We define the one-time signature scheme which we use in our protocol. 


	A one-time signature scheme $\OTS$ is a triple of PPT algorithms $(\keygen,\sign,\verify)$ such that:
	\begin{itemize}
		\item The key generation algorithm $\keygen$ takes in a security parameter $1^\secpar$ and outputs a key pair $(\VK,\SK)$.
		\item The signing algorithm $\sign$ takes as input $\SK$ and a message $m$ and returns a signature $\sigma$, denoted as $\sigma\gets\sign_\SK(m)$.
		\item The verification algorithm $\verify$ takes as input a verification key $\VK$, a message $m$, and a signature $\sigma$ and returns a bit $b$, denoted as $b\gets\verify_\VK(m,\sigma)$.
	\end{itemize}
	We require the scheme to satisfy the following properties:
	\begin{itemize}
		\item \textbf{Correctness}: For all $\secpar\in\Z_{\geq 0}$, if $(\VK,\SK)\rgets\keygen(1^\secpar)$, then for all $m$ and all $\sigma\gets\sign_\SK(m)$, we have $\verify_\VK(m,\sigma)=1$.
		\item \textbf{One-Time Security:} There exists a negligible function $\negl$ such that for every PPT algorithm $\adv$, $$\Pr[\verify_\VK(m,\sigma)=1: (\VK,\SK)\rgets\keygen(1^\secpar), (m,\sigma)\gets\adv^{\sign_\SK(\cdot)}(1^\secpar,\VK)]\leq\negl(\secpar)$$ where $\adv$ makes only a single query to $\sign_\SK(\cdot)$ and $\sigma$ was not an output of $\sign_\SK(m)$.
	\end{itemize}


\subsubsection{Collision-Resistant Hash Functions}

Informally, a function $H$ is collision-resistant if no PPT algorithm can find $x\neq x'$ such $H(x)=H(x')$ with all but negligible probability. Formally, we define a hashing scheme $\mathsf{HF}$ associated with $\mathcal{G}$ to specify (a) a family of key spaces $\mathsf{HF.Keyspace}_{\secpar,\Gamma}$ indexed by $\lambda\in\Z_{\geq 0}$ and $\Gamma\in[S_\secpar]$, and (b) a family of hash functions indexed by $\lambda\in\Z_{\geq 0}$, $\Gamma\in[S_\secpar]$ and $\hk\in[\mathsf{HF.Keyspace}_{\secpar,\Gamma}]$ where each $\mathsf{HF}_\hk^{\lambda,\Gamma}$ maps a tuple $\rho$ of group elements to an element of $\Z_q$. Furthermore, the algorithm that outputs $\mathsf{HF}_\hk^{\lambda,\Gamma}(\rho)$ must be deterministic and polynomial time.

Let $\adv$ be any PPT algorithm. For $\lambda\in\Z_{\geq 0}$, we define the CRH advantage of $\adv$ given $\Gamma$ as $$\advcrh_{\mathsf{HF},\adv}(\secpar | \Gamma):=\Pr[x\neq x'\wedge\mathsf{HF}_\hk^{\lambda,\Gamma}(x)=\mathsf{HF}_\hk^{\lambda,\Gamma}(x'): \hk\rgets\mathsf{HF.Keyspace}_{\lambda,\Gamma}, (x,x')\gets\adv(1^\secpar,\Gamma,\hk)]$$
where $\Gamma\rgets[S_\lambda]$. We say that $\mathsf{HF}$ is collision-resistant if there exists some negligible function $\negl$ such that for all PPT $\adv$ and $\secpar\in\Z_{\geq 0}$, $$\advcrh_{\mathsf{HF},\adv}(\secpar|\Gamma)\leq\negl(\secpar)$$ where $\Gamma\rgets[S_\lambda]$.

\subsection{Cramer-Shoup Encryption Scheme}

Our protocol uses the security of the Cramer-Shoup public-key encryption scheme, which is a scheme proven CCA secure under standard cryptographic assumptions (the Decisional Diffie-Hellman Assumption). We will use a modification of the Cramer-Shoup Scheme that supports \textit{labels} \cite{EC:KatOstYun01, iso}.

	\begin{figure}[tbp]
	\begin{framed}\small
		\textbf{Key Generation:} On input $1^\lambda$ from $\lambda\in\mathbb{Z}_{\geq 0}$, compute
		\begin{align*}
			&\Gamma[\hat{\mathbb{G}}, \mathbb{G}, g, q]\rgets\mathcal{G}(1^\lambda); \hk\rgets\mathsf{HF.Keyspace}_{\lambda,\Gamma}\\
			&w\rgets\mathbb{Z}_q^{*}; x_1,x_2,y_1,y_2,z_1,z_2\rgets\mathbb{Z}_q\\
			&g_1\gets g; g_2\gets g^{w}; c = g_1^{x_1}g_2^{x_2}; d\gets g_1^{y_1}g_2^{y_2}; h\gets g_1^{z_1}g_2^{z_2}
		\end{align*}
		and output the public key $\PK = (\Gamma,\hk,g_1,g_2,c,d,h)$ and the secret key $\SK = (\Gamma, \hk, x_1,x_2,y_1,y_2,z_1,z_2).$
		
		\vspace{4mm}
		
		\textbf{Encryption:} Given $1^\lambda$ for $\lambda\in\mathbb{Z}_{\geq 0}$, a public key $$\PK=(\Gamma,\hk,g_1,g_2,c,d,h)\in[\mathcal{G}_\lambda]\times[\mathsf{HF.Keyspace}_{\lambda,\Gamma}]\times\G^4,$$ a message $m\in G$ and a label $\lab$, compute
		
		\begin{itemize}
			\itemsep=0em
			\item[]\textbf{$\estep{1}$}: $r\rgets\mathbb{Z}_q$;
			\item[]\textbf{$\estep{2}$}: $u_1\rgets g_1^r$;
			\item[]\textbf{$\estep{3}$}: $u_2\rgets g_2^r$;
			\item[]\textbf{$\estep{4}$}: $e'\gets h^r$;
			\item[]\textbf{$\estep{5}$}: $e\gets e'\cdot m$;
			\item[]\textbf{$\estep{6}$}: $\alpha\gets\mathsf{HF}^{\lambda,\Gamma}_{\hk}(\lab, u_1, u_2, e)$;
			\item[]\textbf{$\estep{7}$}: $v\gets (cd^\alpha)^r$;
		\end{itemize}
		and output the ciphertext $\psi = (u_1, u_2, e, v)$.
		
		\vspace{4mm}
		
		\textbf{Decryption:} Given $1^\lambda$ for $\lambda\in\mathbb{Z}_{\geq 0}$, a secret key $$\SK=(\Gamma, \hk, x_1,x_2,y_1,y_2,z_1,z_2)\in[\mathcal{G}_\lambda]\times[\mathsf{HF.Keyspace}_{\lambda,\Gamma}]\times\mathbb{Z}_q^{6},$$
		along with a ciphertext $\psi$ and a label $\lab$, do the following.
		
		\begin{itemize}
			\itemsep=0em
			\item[]\textbf{$\dstep{1}$}: Parse $\psi$ as a $4$-tuple $(u_1,u_2,e,v)\in\G^4$, output $\reject$ and halt if $\psi$ is not of this form.
			\item[]\textbf{$\dstep{2}$}: Test if $u_1, u_2$ and $e$ belong to $\G$; output $\reject$ and halt if this is not the case.
			\item[]\textbf{$\dstep{3}$}: Compute $v'\gets\mathsf{HF}^{\lambda,\Gamma}_{\hk}(\lab,u_1,u_2,e)$.
			\item[]\textbf{$\dstep{4}$}: Test if $v = u_1^{x_1+y_1\alpha}u_2^{x_2+y_2\alpha};$ output $\reject$ and halt if this is not the case.
			\item[]\textbf{$\dstep{5}$}: Compute $e' = u_1^{z_1}u_2^{z_2}$.
			\item[]\textbf{$\dstep{6}$}: Compute $m\gets e\cdot {e'}^{-1}$ and output $m$.
		\end{itemize}
	\end{framed}
	\caption{The Cramer-Shoup Encryption Scheme with Labels.}
	\label{fig:cs03}
	\end{figure}

We provide a formal description of the scheme in \cref{fig:cs03}. Apart from the CCA security of the scheme, we rely on a modified variant of security we term \textit{delayed-reveal-CCA} security, in which the adversary is revealed certain (potentially sensitive) information \textit{after} all decryption queries have been made. This modified security property is crucial to our security proof. We postpone discussion of DR-CCA security to \cref{appendix-a}.

\subsection{UC-PAKE}
	
	We recall the standard UC-PAKE functionality \cite{EC:CHKLM05} in \cref{fig:pake-functionality}. A more detailed explanation can be found in \cite{PKC:RoyXu23}.
	
	\begin{figure}[tbp]
		\begin{framed}\small
			\begin{itemize}
				\item On input $(\NewSession, sid, \party, \party', \pw, \role)$ from $\party$, send $(\NewSession,\allowbreak sid,\allowbreak \party,\allowbreak \party',\allowbreak \role)$ to $\SIM$. Furthermore, if this is the first $\NewSession$ message for $sid$, or this is the second $\NewSession$ message for $sid$ and there is a record $\left<\party', \party, \cdot\right>$, then record $\left<\party, \party', \pw\right>$ and mark it $\fresh$.
				\item On $(\TestPwd, sid, \party, \pw^*)$ from $\SIM$, if there is a record $\left<\party, \party', \pw\right>$ marked $\fresh$, then do:
				\begin{itemize}
					\item If $\pw^* = \pw$, then mark the record $\compromised$ and send ``correct guess'' to $\SIM$.
					\item If $\pw^* \neq \pw$, then mark the record $\interrupted$ and send ``wrong guess'' to $\SIM$.
				\end{itemize}
				\item On $(\NewKey, sid, \party, K^* \in \{0,1\}^\secpar)$ from $\SIM$, if there is a record $\left<\party, \party', \pw\right>$, and this is the first $\NewKey$ message for $sid$ and $\party$, then output $(sid, K)$ to $\party$, where $K$ is defined as follows:
				\begin{itemize}
					\item If the record is $\compromised$, or either $\party$ or $\party'$ is corrupted, then set $K := K^*$.
					\item If the record is $\fresh$, a key $(sid, K')$ has been output to $\party'$, at which time there was a record $\left<\party', \party, \pw\right>$ marked $\fresh$, then set $K := K'$.
					\item Otherwise sample $K \gets \{0,1\}^\secpar$.
				\end{itemize}
				Finally, mark the record $\completed$.
			\end{itemize}
		\end{framed}
		\caption{UC PAKE functionality $\pake$}
		\label{fig:pake-functionality}
	\end{figure}
	
	We briefly describe how adversarial attacks are modelled within the scope of the functionality, which is crucial to our security analysis.
	
	\begin{itemize}
		\item In the case of an \textit{eavesdropping} attack, both sessions $\party$ and $\party'$ are marked as $\fresh$. In this case the protocol execution in the real world is correct, and $\testpwd$ command is sent to $\pake$. In this case, a session key $K$ is selected first \textit{uniformly}, and the other session key $K':=K$. This corresponds to a correct protocol execution without adversary interference.
		\item In the case of an eavesdropping attack with \textit{incorrect password}, it is true that $\pw\neq\pw'$ and no $\testpwd$ is sent. In this case, however, independently random keys $K$ and $K'$ are sampled.
		\item In case of a successful guessing attack, the simulator sends a $\testpwd$ query and receives `correct guess' while the session is marked as $\compromised$. The simulator then submits a key $K^{*}$ (which is output in the real protocol and can be determined from the actions of the adversary) to $\pake$ which sets the key of the party to $K^{*}$. \textit{Note that this is a significant departure from the game-based security definition of PAKE.} In the game-based security definition, all security guarantees are lost in the case of a correct password guess, and the simulation can be terminated (by outputting some target message such as $\bot$). In the UC-PAKE security definition, simulation must continue, and the output of the protocol must be the `detrmined' key $K^{*}$, which may need to be \textit{extracted} from the adversary's messages.
		\item If the session is $\interrupted$, this corresponds to a failed guessing attack by the adversary. In this case, the session keys of $\party$ and $\party'$ are uniform and independent, and the adversary gains no knowledge of either of the session keys.
		\item If a session is $\fresh$ but the corresponding session of $\party'$ is attacked, then the adversary should gain no information about the key of $\party$, and it must be output uniformly at random.
	\end{itemize}
	
	Our proof of security relies heavily on simulating each of the above cases.
	
	
	
%	\begin{figure}
%		\begin{framed}
%			
%		\end{framed}
%	\end{figure}
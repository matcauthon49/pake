\section{UC-Security of KOY in the Algebraic Group Model}

We will show the following theorem.

\begin{theorem}
	The protocol of [KOY] UC-realizes the $\pake$ functionality in the Algebraic Group Model under the SDH and DDH Assumptions.
\end{theorem}

The broad strokes of our proof are as below.

\begin{enumerate}
	\item In game $\game_0'$, the experiment is aborted if any of the following events happen.
	\begin{enumerate}
		\item The client accidentally generates the same $\VK$ twice for different $\sid$s. In this case, the challenger detects this in the message that is output, sends an $\abort$ and ends the protocol.
		\item The server generates $\msg{2}$ twice for different $\sid$s. The challenger detects this in the message that is output, sends an $\abort$ and ends the protocol.
		\item The challenger detects at some point that the adversary has been able to produce a new signed message for some honestly-generated $\VK$ without knowing the corresponding signing key (which it can't, because the $\VK$ is honestly generated -- if it was the one who generated that $\VK$, the adversary is allowed to do whatever it wants with it).
		\item At any given point, the challenger detects that a collision has occured in the hash function. In this case the challenger again sends an $\abort$ and terminates the protocol.
	\end{enumerate}
	This is indistinguishable from the attack $\game_0$ because the probability of any of these events happening is unlikely due to the security of the hash functions and the one-time signature scheme.
	
	\item In game $\game_1$, the challenger changes the way that the $\crs$ is generated. In this case, the $\crs$ is now generated according to the techniques laid out in the Cramer-Shoup $\keygen$ procedure. The distribution of generated values is information-theoretically identical; the only difference is that now the challenger has planted trapdoors into the description of the $\crs$, and it knows the discrete logs of $h, c$ and $d$. The simulator is going to (eventually) use this to extract a password guess.
	
	\item In game $\game_2$, the challenger considers the messages $\msg{i}^{*}$ received from the adversary, and uses the parameters in $\game_1$ to extract a password guess. If the password guess is correct, it extracts the computed session key in the compromised protocol and sets it to be $\sk_i$. Note that the change is purely conceptual; the game is indistinguishable from the previous one.
	
	\item In game $\game_3$, the challenger deals with incoming $\msg{3}^{*}$ queries. It checks the $\sid$ of the message and in case that $\msg{1} = \msg{1}^{*}$, it sets $\sk_\server:=\sk_\user$ for the same $\sid$, which was already previously computed (at the time that $\msg{3}$ was output).  Note that in this case $\msg{3}$ \textit{has} to be output by the client -- the only situation in which this doesn't happen is if the environment was able to forge some $(K,\mathsf{sig})$, but $\game_0'$ necessitates abort in the case that this occurs. Thus, the game is valid, and furthermore is indistinguishable because this is, again, only a conceptual change.
	
	\item In game $\game_4$, the challenger again deals with incoming $\msg{3}^{*}$ queries. It again checks the $\sid$ of the message and in the case that $\msg{1}\neq\msg{1}^{*}$ and furthermore the message is an invalid Cramer-Shoup ciphertext of $\pw_\user$, the challenger outputs a session key $\sk_\server$ \textit{at random}. The proof of indistinguishability proceeds from an information-theoretic argument.
	
	\item In game $\game_5$, the challenger replaces the server's Cramer-Shoup encryption of $\pw_\user$ with an encryption of some $g^k\notin\mathcal{PW}$. The proof of indistinguishability follows from the fact that any environment which can differentiate between the two games can also be used as a subroutine to break the CCA-security of Cramer-Shoup.
	
	\item In game $\game_6$, the challenger considers $\msg{2}^{*}$. If $\msg{2}^{*}=\msg{2}$, then the challenger sets $\sk_\user$ to be random. The proof of this game is similar to that of $\game_3$.
	
	\item In game $\game_7$, the challenger considers $\msg{2}^{*}$ again, and the case in which $\msg{2}^{*}\neq\msg{2}$ but the message does not contain a valid password guess, it sets $\sk_\user$ to be random as well. The proof is again similar.
	
	\item In game $\game_8$, the challenger replaces the client's Cramer-Shoup encryption of $\pw_\user$ with an encryption of some $g^k\notin\mathcal{PW}$. The proof of indistinguishability follows from the DR-CCA security of Cramer-Shoup, which is discussed in \cref{appendix-a}.
	
	\item Game $\game_9$ is the ideal functionality. We argue that the structure of $\game_8$ can be interpreted as the ideal functionality with no security loss.

	
\end{enumerate} 


\subsection{Notation}

We describe the notation we have to use. Let $Z_i$ denote the event that $\mathcal{Z}$ outputs $1$ in game $G_i$. We need to show that $Z_0\cong Z_9$, which corresponds to the fact that the real world attack is indistinguishable from the ideal world attack.

We refer to the system of $\party_0$ and $\party_1$

Let $\msg{i}$ denote the $i$th message output by the

\subsection{Proof of UC-Security}

We now provide a formal description of the proof.

\textbf{Game $\game_0$.} This game corresponds to the real-world attack.

\textbf{Game $\game_0'$.} In this game, the challenger interacts with the adversary as before, except we introduce an additional abort clause. If this clause is activated, then the challenger sends a message $\abort$ and ends the protocol. The abort clause is activated in the case of the following situations:
\begin{enumerate}
	\item
\end{enumerate}
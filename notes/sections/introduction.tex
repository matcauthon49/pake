\section{Introduction}
\subsection{Protocol Description}
\xjy{Insert the description of KOY (revised to fit the UC formality) here.} \nam{I've made edits, clarified the scheme.}

We provide a high-level description of the protocol along with a sketch of the security proof below. Let $\G$ be an algebraic group in which DDH is believed to be hard and define $\crs := (g_1, g_2, h, c, d)$ to be uniformly sampled from $\G^5$; we set this to be the common random string. Note that $\crs$ can be interpreted to be a degenerate Cramer-Shoup public key with an unknown secret key.

\begin{itemize}
	\item $\user$ begins by generating a pair of keys $(\VK,\SK)$ for a one-time signature scheme and encrypts its password $\pw$ with label $\VK$ using the Cramer-Shoup public key embedded in the CRS. Let $H$ be a collision-resistant hash function and $r_1$ be the randomness used during encryption. The resulting ciphertext consists of four group elements $\ct_1 := (A,B,C,D) = (g_1^{r_1}, g_2^{r_1}, h^{r_1} \cdot\pw, (cd^\alpha)^{r_1})$ where $\alpha = H(\VK, A, B, C)$. $\user$ then sends $\msg{1} := (\VK, \ct_1)$ to $\server$. Note that $A$, $B$, $C' = C/\pw$, $D$ are all of the form $g^{r_1}$ where $g$ is some group element that $\server$ can compute, so they also serve as a message in a Diffie--Hellman-like protocol.
	
	\item Upon receiving $(\VK,A,B,C,D)$, the $\server$ samples its `Diffie--Hellman exponents' $(x_2, y_2, z_2, w_2)$, and computes $E = g_1^{x_2}g_2^{y_2}h^{z_2}(cd^\alpha)^{w_2}$.\footnote{Note that $\alpha = H(\VK|A|B|C)$, so $\server$ must wait for $\user$'s message before proceeding.} Furthermore, $\server$ encrypts $\pw$ with label $(\msg{1}, E)$ using Cramer--Shoup encryption as done by $\user$ in $\msg{1}$. Let $r_2$ be the randomness used in encryption. The resulting ciphertext is $\ct_2:=(F,G,I,J) = (g_1^{r_2}, g_2^{r_2}, h^{r_2} \cdot\pw, (cd^\beta)^{r_2})$. $\server$ sends $\msg{2} = (E,\ct_2)$ to $\user$.
	
	\item Symmetrically, $\user$ upon receiving $(E,\ct_2)$ samples its own `Diffie--Hellman exponents' $(x_1,y_1,z_1,w_1)$ and computes $K = g_1^{x_1}g_2^{y_1}h^{z_1}(cd^\beta)^{w_1}$. It signs the protocol transcript $\st = \sign_\SK(\msg{1}|\msg{2}|K)$, and sends $\msg{3} = (K, \st)$ to the server.
	
	\item $\server$ verifies the signature $\st$ and aborts if it is invalid. Otherwise the session key $\sk$ is defined as the product of
	\[
	X_1 = E^{r_1} = A^{x_2}B^{y_2}(C/\pw)^{z_2}D^{w_2}
	\]
	and
	\[
	X_2 = K^{r_2} = F^{x_1}G^{y_1}(I/\pw)^{z_1}J^{w_1}
	\]
	$\user$ can compute $X_1$ as $E^{r_1}$ and $X_2$ as $F^{x_1}G^{y_1}(I/\pw)^{z_1}J^{w_1}$, whereas $\server$ can compute $X_1$ as $K^{r_2}$ and $X_2$ as $F^{x_1}G^{y_1}(I/\pw)^{z_1}J^{w_1}$.
\end{itemize}

\paragraph{Security.}
To perform authentication, $\user$ and $\server$ need to (implicitly) prove to each other that they know $\pw$. This is achieved as follows. Note that $X_1 = E^{r_1} = A^{x_2}B^{y_2}(C/\pw)^{z_2}D^{w_2}$ has the following property: given $E$ (but not $x_2,y_2,z_2,w_2$), if $A|B|C|D$ is a valid encryption of $\pw$, then knowing the randomness $r_1$ is sufficient for computing $X_1$; otherwise $X_1$ is a uniformly random group element.\footnote{Using the terminology of SPHF: $(x_2,y_2,z_2,w_2)$ is the hash key and $E$ is the corresponding projection key; the function is defined as $\mathsf{Hash}_{(x_2,y_2,z_2,w_2)}(m) = A^{x_2}B^{y_2}(C/m)^{z_2}D^{w_2}$.} Therefore, an adversarial user that does not know $\pw$, is not able to come up with a valid $A|B|C|D$, so $X_1$ is uniformly random in the adversary's view (and so is $\sk = X_1X_2$); a symmetric argument can be made for an adversarial server. In the man-in-the-middle setting, an adversary can attempt to generate a valid ciphertext after seeing another valid ciphertext from the honest user/server, so we need the encryption scheme to be non-malleable.

The one-time signature scheme effectively forces the adversary to pass $\msg{2}$ and $\msg{3}$ without modification as long as it passes $\msg{1}$. This is to prevent a man-in-the-middle adversary from gaining information by passing all ciphertexts but modifying the rest of the messages. Specifically, (if the signature scheme is removed) consider an adversary that passes $\msg{1} = A|B|C|D$, changes $\msg{2} = E|F|G|I|J$ to $E^{\frac{1}{2}}|F|G|I|J$, and changes $\msg{3} = K$ to $K^2$; this would cause $\sk_S = \sk_U^2$. In other words, the adversary (that does not know the password) causes the two parties' session keys to be unequal but correlated, which is not allowed by the security of PAKE. Furthermore, to prevent the adversary from plugging in its own verification key (and thus knowing the corresponding secret key), $\VK$ is included in the hash that produces $\alpha$. In this way if the adversary changes $\VK$ while keeping the ciphertext $A|B|C|D$, the ciphertext would become invalid.

\subsection{Technical Overview}

In this section, we provide a high-level explanation of why the KOY protocol is insecure in the UC framework, and how the AGM circumvents the difficulty for the UC simulator.

\paragraph{UC-insecurity of KOY.}
Our attack relies on an adversary $\adv$ that completely disregards the presence of $\server$ and instead interacts with $\user$ while executing $\server$'s algorithm on its own. In particular, once the protocol is initiated by $\user$, $\adv$ assumes the role of the server (discarding the actual server in the process, which plays no part in the protocol) and receives $\msg{1} = \mathsf{VK}|A|B|C|D$. After this, $\adv$ runs the server's algorithm on $\user$'s password $\pw$ (i.e., we assume that $\adv$ makes a correct password guess) and computes $\msg{2} = E|F|G|I|J$. $\user$ then runs its session-key generating algorithm and outputs its session key $\sk = E^{r_1}F^{x_1}G^{y_1}(I/\pw)^{z_1}J^{w_1}$. At this point, $\adv$ (and $\env$) have all the information they need to run the server's session-key generating algorithm locally, which computes a session key equal to $\sk$ generated by $\user$.

To see why a simulator $\simulator$ cannot simulate this adversary, we attempt an ideal-world execution and pinpoint where it fails. Since $\simulator$ is allowed to choose $\crs=(g_1, g_2, h, c, d)$, it can sample $g_1$ at random and set $h$ such that $h=g_1^{\ell}$ --- in other words, $\simulator$ chooses the ``CRS trapdoor'' $\ell = \log_{g_1} h$. After receiving the $\newsession$ command from $\pake$, $\simulator$ must simulate $\user$'s first message $\msg{1}$. Since $\simulator$ does not know the password, at this point it must (effectively) guess some $\pw'$ at random; that is, in $\msg{1}$, $C=h^{r_1}\cdot\pw'$ where $\pw'$ can be no better than a random password sampled from the dictionary. ($C$ is indistinguishable from the correct value due to the security of Cramer--Shoup encryption.) After $\env$ responds with $\msg{2} = E|F|G|I|J$, since $F = g_1^{r_2}$ and $I = h^{r_2} \cdot \pw$, $\simulator$ can extract $\pw$ as $I/F^\ell$. Once this has been done, $\simulator$ can send a $\testpwd$ command to $\pake$ on the correct $\pw$; $\pake$ would mark the $\user$ session $\compromised$ and thus allow $\simulator$ to choose $\user$'s session key $\sk$ (which has to be consistent with the session key $\user$ computes in the real world).\footnote{A $\testpwd$ \textit{must} be run, since we require that $\msg{1}$ and $\msg{2}$ together with the randomness of the $\user$ and $\adv$ together determine $\sk$; allowing the simulation to proceed without a $\testpwd$ would result in $\pake$ outputting a uniformly random key.} This is where the game-based security and UC-security of PAKE diverge: in game-based security, all security guarantees are considered lost (and the simulation of the game can stop) once the adversary guesses the correct password; whereas in UC-security the simulation has to continue. The problem here is that \emph{even knowing the correct password $\pw$, $\simulator$ still cannot determine what $\sk$ should be}.

Recall that $\sk$ is the product of
\[
X_1 = E^{r_1} = A^{x_2}B^{y_2}(C/\pw)^{z_2}D^{w_2}
\]
and
\[
X_2 = K^{r_2} = F^{x_1}G^{y_1}(I/\pw)^{z_1}J^{w_1}
\]
Computing $X_2$ is not a problem for $\simulator$, since $\msg{2} = E|F|G|I|J$ is provided to $\simulator$ directly from the environment; $\simulator$ chose $x_1, y_1, z_1, w_1$ on its own; and $\simulator$ has extracted $\pw$. However, $\simulator$ is not able to compute $X_1$. At first glance, computing $X_1 = E^{r_1}$ might appear feasible as $\simulator$ received $E$ as part of $\msg{2}$ and sampled $r_1$ before sending $\msg{1}$. However, the problem is that \emph{the password guess $\pw'$ that $\simulator$ uses while generating $\msg{1}$ is likely incorrect}; as a result, $E^{r_1}$ that $\simulator$ computes is actually equal to $A^{x_2}B^{y_2}(C/\pw')^{z_2}D^{w_2}$, whereas the correct value should be $A^{x_2}B^{y_2}(C/\pw)^{z_2}D^{w_2}$. This means that $\simulator$ must know $(\pw/\pw')^{z_2}$ in order to compute the correct $\sk$, which is infeasible unless $\pw'=\pw$ (whose probability is $1/\mathcal{|D|}$).

\paragraph{Simulating the session key in AGM.}
% By examining the game-based security proof of the KOY protocol, one can see that the above attack is essentially the only scenario where the UC-security can be broken; in particular, the cases where the adversary assumes the role of the user, as well as the adversary makes an incorrect password guess (by sending the Cramer--Shoup ciphertext of some $\pw^* \neq \pw$, or sending some group elements that are not a Cramer--Shoup ciphertext), can be simulated in UC without any issue. 

Our next critical observation is that the session key in the above attack can be simulated if we resort to the AGM, as the simulator $\simulator$ can extract $x_2, y_2, z_2, w_2$ from an algebraic environment. In more detail, suppose $\env$ runs the above attack and sends $E$ as part of $\msg{2}$. At this point all group elements that $\env$ has seen are $g_1,g_2,h,c,d$ from $\crs$, $A,B,C,D$ from $\msg{1}$, and $\pw$. An algebraic $\env$ must ``explain'' how $E$ is computed; for now let's ignore $A,B,C,D,\pw$ and assume $\env$ computes
\[
E = g_1^{x_2}g_2^{y_2}h^{z_2}(cd^\alpha)^{w_2}
\]
In the real world $\user$ would compute $X_1 = E^{r_1}$, which is equal to $A^{x_2}B^{y_2}(C/\pw)^{z_2}D^{w_2}$. In the ideal world, as explained above, $\simulator$ cannot compute $X_1$ as $E^{r_1}$ since it chose the wrong $\pw'$ with high probability while generating $A,B,C,D$; however, after extracting the correct $\pw$ from $\msg{2}$, $\simulator$ can still compute $X_1$ as $A^{x_2}B^{y_2}(C/\pw)^{z_2}D^{w_2}$, since now it sees the algebraic coefficients $x_2,y_2,z_2,w_2$ from $\env$. In this way $\simulator$ generates $X_1$ that is indistinguishable from the real world.

In the general case, suppose $\env$ computes
\[
E = g_1^{x_2}g_2^{y_2}h^{z_2}c^{w_2}d^{v_2}A^{x_2'}B^{y_2'}C^{z_2'}D^{w_2'}\pw^{p_2}
\]
In the real world $\user$ would compute
\begin{align*}
	X_1 &= E^{r_1} \\
	&= g_1^{r_1x_2} g_2^{r_1y_2} h^{r_1z_2} c^{r_1w_2} d^{r_1v_2} A^{r_1x_2'} B^{r_1y_2'} C^{r_1z_2'} D^{r_1w_2'} \pw^{r_1p_2} \\
	&= g_1^{r_1x_2} g_2^{r_1y_2} \left(\frac{C}{\pw}\right)^{z_2} c^{r_1w_2} d^{r_1v_2} A^{r_1x_2'} B^{r_1y_2'} C^{r_1z_2'} D^{r_1w_2'} \pw^{r_1p_2}
\end{align*}	
Again, in the ideal world $\simulator$ can compute $X_1$ according to the last equation; the key point is that given $A|B|C|D$, the only place where the ``wrong'' password $\pw'$ is used lies in $h$, so we only need to ``correct'' $h^{r_1}$ from $C/\pw'$ to $C/\pw$. A difference is that now the expression of $X_1$ involves the Cramer--Shoup randomness $r_1$, which is not a problem for $\simulator$ as $\simulator$ chose $r_1$ itself while sending $\msg{1}$. (Of course, we have $A = g_1^{r_1}$, so the $g_1^{r_1x_2} A^{r_1x_2'}$ part can be rewritten as $A^{x_2+r_1x_2'}$, and so on. But this simplification is not necessary.)

\paragraph{Further subtleties in UC.}
As we have just seen, one critical difference between game-based security and UC-security is that in UC indistinguishability between the real view and the simulated view must remain \emph{even when the environment sees the key of a successfully attacked session}, whereas in the game-based setting the adversary simply wins (and the simulator can ``give up'' on simulating the session key) once the attack on a session is successful. While the session key itself can be simulated in the AGM, this poses another potential issue that is more subtle: after seeing the session key, the environment might go back and check the validity of previous protocol messages using this information.

In more detail, assume again that the adversary interacts with $\user$ by running $\server$'s algorithm on $\user$'s password $\pw$. When $\user$'s session completes, the environment $\env$ sees $\user$'s session key $\sk_U = X_1X_2$ and the last message $\msg{3} = K|\sigma$. Since $\env$ can compute $X_2$ as $K^{r_2}$ (since $r_2$ is chosen by $\env$ itself), it can recover $X_1 = E^{r_1}$ (where $r_1$ is the randomness used in $\msg{1}$). Recall that in the ideal world the Cramer--Shoup ciphertext $A|B|C|D$ generated by the simulator $\simulator$ is an encryption of some $\pw'$ that is unlikely to be the ``correct'' $\pw$, so $\env$ must not be able to detect this fact even after seeing $E^{r_1}$. In other words, (very roughly) \emph{Cramer--Shoup must be secure even if the adversary sees $E^{r_1}$ for some $E$ of its choice}.

Below we analyze two simple attacks using this strategy:
\begin{enumerate}
	\item Say $\env$ chooses $E = A = g_1^{r_1}$; then $E^{r_1} = g_1^{r_1^2}$. If 2-DL is easy in the group\footnote{The 2-DL problem is: given $(g^x,g^{x^2})$ for $x \gets \mathbb{Z}_q$, compute $x$. We do not know the relative hardness between 2-DL and DDH, and it has been shown that 2-DL and CDH are separate in the AGM \cite{C:BauFucLos20}.}, then $\env$ can recover $r_1$ after $\user$'s session completes, and check if $C = h^{r_1} \cdot \pw$. In the real world this equation holds, whereas in the ideal world it does not if $\simulator$ chose the ``wrong'' $\pw' \neq \pw$ to encrypt while simulating $\msg{1}$. In fact, it seems that for the KOY protocol to be UC-AGM-secure, we need the \xjy{FILL IN} assumption. \xjy{I think the assumption should be Decisional Square Diffie--Hellman (DSDH), which says that given $g^x$, it is hard to distinguish $g^{x^2}$ from random. Note that this implies 2-DL (given $(g^x,Y)$ where $Y$ is either $g^{x^2}$ or random, the DSDH solver can feed $(g^x,Y)$ to the 2-DL solver and distinguish by observing whether the 2-DL solver wins or not)}
	\item Say $\env$ chooses $E = c$; then $E^{r_1} = c^{r_1}$. But Cramer--Shoup is obviously \emph{not} CCA-secure if the adversary additionally sees $c^{r_1}$. What saves us here is that $\env$ sees $E^{r_1}$ \emph{only at the end of $\user$'s session}. Indeed, the reduction $\red$ to the security of Cramer--Shoup roughly works as follows:
	\begin{enumerate}
		\item $\red$ embeds the challenge ciphertext as $\msg{1}$, $\user$'s first message intercepted by $\adv$;
		\item When $\adv$ sends $\msg{1}^*$ to $\server$ and $\msg{2}^*$ to $\user$, $\red$ needs to query the decryption oracle to extract the password guesses contained in these two messages;
		\item Finally, if the password guess in $\msg{2}^*$ is correct, $\red$ needs to simulate $\sk_U$. (This step is not needed in the game-based proof.)\footnote{We omit the remaining steps which are to simulate $\msg{3}$, and on $\msg{3}^*$ simulate $\server$'s session key $\sk_S$, as they are inconsequential to our main point.}
	\end{enumerate}
	$c^{r_1}$ is needed only in step (c), which happens after all decryption oracle queries have been made. Therefore, we only need Cramer--Shoup to remain CCA-secure if the adversary \emph{is restricted to making all decryption oracle queries before learning $c^{r_1}$}. We will show that Cramer--Shoup indeed satisfies this property in \xjy{FILL IN} \xjy{I think this should work, but it needs to be double checked}
\end{enumerate}
By inspecting the game-based security proof of the KOY protocol, one can see that the above attacks are essentially the only scenarios where the UC-security might be broken, and the security argument for all other cases (including the adversary sending a message that contains an incorrect password guess, or modifying the signature) is essentially identical to the game-based proof. \xjy{I hope so...}
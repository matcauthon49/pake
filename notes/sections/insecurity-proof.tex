
\section{Proof of UC-Insecurity}

\begin{theorem}
	Assuming the hardness of fixed-CDH, the protocol of [KOY] does not UC-realize $\mathcal{F}_\mathsf{pake}$ in the $\mathcal{F}_{\mathsf{crs}}$-hybrid model.
\end{theorem}

\begin{proof}
	
	Consider the environment $\env$ in \Cref{fig:adv} and the dummy adversary. It follows from the correctness of the protocol that in the real-world protocol execution $\env$ always outputs $1$, since the algorithm of $\env$ and $\adv$ is the same as that of an honest server. At a high level, we will show that any simulator that successfully simulates the protocol against $\env$ in the ideal world can be used to solve arbitrary instances of fixed-CDH.
	
	\begin{figure}[ht!]
		\begin{framed}
			% [
			%			linecolor=black,
			%			linewidth=1pt,
			%			roundcorner=5pt,
			%			backgroundcolor=white,
			%			userdefinedwidth=\textwidth,
			%			]
			\vspace{2mm}
			\textbf{\underline{Environment $\env$:}}
			\begin{enumerate}\setcounter{enumi}{-1}
				\item $\env$ receives the CRS $(g_1, g_2, h, c, d)$.
				\item $\env$ selects $\pw\xleftarrow{\$}\mathcal{PW}$, where $\mathcal{PW}\subseteq\mathbb{G}$ is the password dictionary. It then sends $(\newsession,\mathsf{sid},\user,\server,\pw)$ to $\user$.
				\item $\env$ receives $\msg{1} = \mathsf{sid}|\mathsf{VK}|A|B|C|D$ from $\adv$ and samples $x_2, y_2, z_2, w_2, r_2\xleftarrow{\$}\mathbb{Z}_q$. It then sets 
				\begin{align*}
					\alpha'&:=H(A|B|C|D)\\
					E &:= g_1^{x_2}g_2^{y_2}h^{z_2}(cd^{\alpha'})^{w_2}\\
					F &:= g_1^{r_2}\\
					G &:= g_2^{r_2}\\
					I &:= h^{r_2}\cdot\pw\\
					\beta &:= H(\msg{1}|E|F|G|I)\\
					J &:= (cd^{\beta})^{w_2}
				\end{align*}
				and instructs $\adv$ to send $\msg{2} := \mathsf{sid}|E|F|G|I|J$ to $\user$. 
				\item $\env$ receives $\msg{3}=\mathsf{sid}|K|\sigma$ from $\adv$ and $(\mathsf{sid}, \sk)$ from $\user$.
				\item $\env$ sets $C':=C/\pw$ and then checks if $\mathsf{Vrfy}_{\mathsf{VK}}(\msg{1}|\msg{2}|K,\sigma)=1$. If yes, it computes $\sk_S:=A^{x_2}B^{y_2}(C')^{z_2}D^{w_2}K^{r_2}$ and outputs $1$ if $\sk_S=\sk$. If either of the two checks fails, it outputs $0$.
			\end{enumerate}
			\vspace{2mm}
		\end{framed}
		\caption{Our Setup.}
		\label{fig:adv}
	\end{figure}
	
	Assume that there exists a negligible function $\varepsilon:=\varepsilon(\lambda)$ such that there exists a ``successful'' simulator $\simulator$ for which $\env$ outputs $1$ with probability $1-\varepsilon$ in the ideal world. First, assume that CDH is hard over $(\mathbb{G}, p, h)$. Consider reduction $\mathcal{R}$ that runs the simulator $\simulator$ as follows (note that $\mathcal{R}$ plays the role of the environment $\env$, the PAKE functionality $\pake$, and the dummy parties $\user$ and $\server$ combined):
	
	% \nam{I'm not completely convinced if this is the best way to write this; the proof is correct in the essentials, but I think this needs to be rewritten since I'm not sure what formal part $\mathcal{R}$ is playing here...}
	
	\begin{enumerate}\setcounter{enumi}{-1}
		\item $\mathcal{R}$ receives $\mathsf{crs}=(g_1, g_2, h, c, d)$ from $\simulator$, outputs $h$ to its challenger, and receives $(h^a, h^b)$ where $a,b\rgets\mathbb{Z}_p$. 
		
		%\xjy{I moved ``receiving the CRS'' as the 0th step of the formal description of the reduction (and the environment)} \xjy{Is the group order $p$ or $q$? It is called $p$ in this step but $q$ in the second step}
		
		\item $\mathcal{R}$ sends $(\newsession,\mathsf{sid}, \user, \server)$ to $\simulator$.
		\item $\mathcal{R}$ waits to receive $\msg{1} = \mathsf{sid}|\mathsf{VK}|A|B|C|D$ in response (as the first message from $\user$ to $\server$). It then sets $\pw:=C/h^b$ and samples $x_2,y_2,w_2,r_2\rgets\mathbb{Z}_q$, setting
		\begin{align*}
			\alpha'&:=H(PIDs|A|B|C|D)\\
			E &:= g_1^{x_2}g_2^{y_2}h^{a}(cd^{\alpha'})^{w_2}\\
			F &:= g_1^{r_2}\\
			G &:= g_2^{r_2}\\
			I &:= h^{r_2}\cdot\pw\\
			\beta &:= H(\msg{1}|E|F|G|I)\\
			J &:= (cd^{\beta})^{w_2}
		\end{align*}
		i.e., the same computation as that of the honest server with the special choice of $\pw = C/h^b$ and $z_2 = a$, and sends $\msg{2}:=\mathsf{sid}|E|F|G|I|J$ to $\simulator$.
		\item $\mathcal{R}$ receives $\msg{3}=\mathsf{sid}|K|\sigma$ (as the second message from $\user$ to $\server$) and $(\mathsf{sid},\sk)$ from $\simulator$ (as $\user$'s output to $\env$), and checks if $\mathsf{Vrfy}_\mathsf{VK}(\msg{1}|\msg{2}|K,\sigma)=1$. If not, $\mathcal{R}$ aborts. Otherwise it calculates 
		$$h'=\frac{\sk}{A^{x_2}B^{y_2}D^{w_2}K^{r_2}}.$$
		\item $\mathcal{R}$ outputs $h'$.		
	\end{enumerate}
	
	Note that $\simulator$'s view while interacting with $\mathcal{R}$ is identical to $\simulator$'s view in the ideal world with environment $\env$ in \Cref{fig:adv}; the difference is that $\env$ samples $\pw$ and $z_2$ on its own, whereas $\mathcal{R}$ sets $\pw = C/h^b$ and $z_2 = a$ --- which cannot be detected by $\simulator$. Let $C' = C/\pw = h^b$ and
	$$\sk_S = A^{x_2}B^{y_2}(C')^{z_2}D^{w_2}K^{r_2}=A^{x_2}B^{y_2}(h^b)^{a}D^{w_2}K^{r_2}$$
	as what $\env$ would compute in its step 4; $\env$ outputs 1 if and only if $\sk_S = \sk$, so by our assumption on $\simulator$, in $\mathcal{R}$'s interaction with $\simulator$, $\sk_S = \sk$ with probability $1-\varepsilon$. But this gives $h' = h^{ab}$ with probability $1-\varepsilon$, i.e., $\mathcal{R}$ wins with probability $1-\varepsilon$, contradicting the hardness of fixed-CDH.
\end{proof}

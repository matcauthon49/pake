\documentclass[12pt,a4paper]{article}
\usepackage[margin=1in, includeheadfoot]{geometry}

\usepackage[dvipsnames]{xcolor}
\usepackage{amsmath}
\usepackage{amssymb}
\usepackage{amsthm}
%\usepackage{makeidx}
\usepackage{thmtools}
\usepackage{graphicx}
\usepackage{hyperref}
\usepackage{tikz}
\usepackage{algpseudocode}
\usepackage{multirow}
\usepackage{framed}
\usepackage{cleveref} % allow for \cref command


\usetikzlibrary{positioning}

\usepackage{thmtools}
%\usepackage[skip=6pt]{parskip}

\usepackage{fancyhdr}
\pagestyle{fancy}

\fancyhead[L]{\textbf{KOY is not UC-Secure}}
\fancyhead[R]{\thepage}
\fancyfoot{}

\setlength\parindent{0pt}

\hypersetup{
	colorlinks=true,
	linkcolor=red,
	filecolor=red,  
	citecolor=red,    
	urlcolor=blue,
	pdftitle={Games on Graphs \#1},
	pdfpagemode=FullScreen,
}

% \usepackage[framemethod=tikz]{mdframed}

\declaretheorem[name=Theorem, numberwithin=section]{theorem}
\declaretheorem[name=Lemma, numberwithin=section]{lemma}
\declaretheorem[name=Inequality, numberwithin=section]{inequality}
\declaretheorem[name=Definition, numberwithin=section]{definition}

\usepackage[english]{babel}
\title{\textbf{KOY is not UC-Secure}}
\author{Naman Kumar}

%------- General Macros

\newcommand{\np}{\mathbf{NP}}
\newcommand{\p}{\mathbf{P}}
\newcommand{\ip}{\mathbf{IP}}
\newcommand{\pspace}{\mathbf{PSPACE}}
\newcommand{\am}{\mathbf{AM}}
\newcommand{\ma}{\mathbf{MA}}
\newcommand{\F}{\mathbb{F}}
\newcommand{\E}{\mathbb{E}}
\newcommand{\rel}{\mathcal{R}}
\newcommand{\lang}{\mathcal{L}}
\newcommand{\query}{\mathcal{Q}}
\newcommand{\prover}{\mathcal{P}}
\newcommand{\verifier}{\mathcal{V}}
\newcommand{\decision}{\mathcal{D}}
\newcommand{\simulator}{\mathcal{S}}
\newcommand{\sat}{\mathsf{SAT}}
\newcommand{\nc}{\mathbf{NC}}
\newcommand{\hvzk}{\mathbf{HVZK}}
\newcommand{\zk}{\mathbf{ZK}}
\newcommand{\pzk}{\mathbf{PZK}}
\newcommand{\czk}{\mathbf{CZK}}
\newcommand{\szk}{\mathbf{SZK}}
\newcommand{\pc}{\mathsf{pc}}
\newcommand{\vc}{\mathsf{vc}}
\newcommand{\vr}{\mathsf{vr}}
\newcommand{\poly}{\mathsf{poly}}
\newcommand{\add}{\mathsf{add}}
\newcommand{\mul}{\mathsf{mul}}
\newcommand{\view}{\mathsf{View}}
\newcommand{\trace}{\textsc{Trace}}
\newcommand{\bpp}{\mathbf{BPP}}
\newcommand{\hilbert}{\mathcal{H}}
\newcommand{\ket}[1]{\left|#1\right\rangle}
\newcommand{\bra}[1]{\left\langle#1\right|}
\newcommand{\braket}[2]{\langle#1|#2\rangle}
\newcommand{\puncture}{\mathsf{Puncture}}

%---------- Macros

\newcommand{\env}{\mathcal{Z}}
\newcommand{\adv}{\mathcal{A}}
\newcommand{\pake}{\mathcal{F}_{\mathsf{PAKE}}}
\newcommand{\user}{\mathsf{User}}
\newcommand{\sk}{\mathsf{sk}}
\newcommand{\pw}{\mathsf{pw}}
\newcommand{\VK}{\mathsf{VK}}
\newcommand{\SK}{\mathsf{SK}}
\newcommand{\crs}{\mathsf{crs}}
\newcommand{\newsession}{\mathsf{NewSession}}
\newcommand{\testpwd}{\mathsf{TestPwd}}
\newcommand{\compromised}{\mathsf{compromised}}
\newcommand{\server}{\mathsf{Server}}
\newcommand{\msg}[1]{\mathsf{msg}_{#1}}
\newcommand{\rgets}{\xleftarrow{\$}}

\def\xjy#1{\textcolor{blue}{Jiayu: #1}}
\def\nam#1{\textcolor{red}{Naman: #1}}

\begin{document}
	\maketitle
	
	This document contains a proof that the PAKE protocol of Katz, Ostrovsky and Yung (2001) is not UC-secure.

\section{UC-(In)Security of the KOY Protocol}
\subsection{Protocol Description}
\xjy{Insert the description of KOY (revised to fit the UC formality) here}

We provide a high-level description of the protocol, as well as why it is secure. The CRS is the public key of the Cramer--Shoup encryption scheme, $(g_1, g_2, h, c, d)$; we stress that unlike in the PKI setting, the protocol does not involve parties sending their public keys to each other, and no one knows the corresponding secret key. To execute the protocol,
\begin{itemize}
  \item $\user$ generates a pair of keys for a one-time signature scheme $(\VK,\SK)$, and encrypts its password $\pw$, with label $\VK$, using Cramer--Shoup encryption (let $r_1$ be the randomness). The resulting ciphertext is four group elements $(A,B,C,D) = (g_1^{r_1}, g_2^{r_1}, h^{r_1} \cdot\pw, (cd^\alpha)^{r_1})$, which $\user$ sends to $\server$ together with $\VK$. Note that $A$, $B$, $C' = C/\pw$, $D$ are all in the form of $g^{r_1}$ where $g$ is some group element that $\server$ can compute, so they also serve as a message in a Diffie--Hellman-like protocol.
  \item $\server$, upon receiving $\VK,A,B,C,D$, samples its ``Diffie--Hellman exponents'' $x_2,y_2,z_2,w_2$, and computes $E = g_1^{x_2}g_2^{y_2}h^{z_2}(cd^\alpha)^{w_2}$.\footnote{Note that $\alpha = H(\VK|A|B|C)$, so $\server$ must wait for $\user$'s message before proceeding.} Furthermore, $\server$ encrypts $\pw$, with label $\msg{1}$ and $E$, using Cramer--Shoup encryption (let $r_2$ be the randomness). The resulting ciphertext is $(F,G,I,J) = (g_1^{r_2}, g_2^{r_2}, h^{r_2} \cdot\pw, (cd^\beta)^{r_2})$, which $\server$ sends to $\user$ together with $E$.
  \item Symmetrically, $\user$, upon receiving $E,F,G,I,J$, samples its ``Diffie--Hellman exponents'' $x_1,y_1,z_1,w_1$, and computes $K = g_1^{x_1}g_2^{y_1}h^{z_1}(cd^\beta)^{w_1}$. $\user$ then signs the protocol transcript $\msg{1}|\msg{2}|K$ using $\SK$, and sends $K$ together with the signature to $\server$.
  \item $\server$ verifies the signature, and aborts if it is invalid. Otherwise the session key $\sk$ is defined as the product of
      \[
      X_1 = E^{r_1} = A^{x_2}B^{y_2}(C/\pw)^{z_2}D^{w_2}
      \]
      and
      \[
      X_2 = K^{r_2} = F^{x_1}G^{y_1}(I/\pw)^{z_1}J^{w_1}
      \]
      $\user$ can compute $X_1$ as $E^{r_1}$ and $X_2$ as $F^{x_1}G^{y_1}(I/\pw)^{z_1}J^{w_1}$, whereas $\server$ can compute $X_1$ as $K^{r_2}$ and $X_2$ as $F^{x_1}G^{y_1}(I/\pw)^{z_1}J^{w_1}$.
\end{itemize}

\paragraph{Security.}
To perform authentication, $\user$ and $\server$ need to (implicitly) prove to each other that they know $\pw$. This is achieved as follows. Note that $X_1 = E^{r_1} = A^{x_2}B^{y_2}(C/\pw)^{z_2}D^{w_2}$ has the following property: given $E$, if $(A,B,C,D)$ is a valid encryption of $\pw$, then knowing the randomness $r_1$ is sufficient for computing $X_1$; otherwise $X_1$ is a uniformly random group element.\footnote{Using the terminology of SPHF: $(x_2,y_2,z_2,w_2)$ is the hash key and $E$ is the corresponding projection key; the function is defined as $\mathsf{Hash}_{(x_2,y_2,z_2,w_2)}(m) = A^{x_2}B^{y_2}(C/m)^{z_2}D^{w_2}$.} Therefore, an adversarial user that does not know $\pw$, is not able to come up with a valid $(A,B,C,D)$, so $X_1$ is uniformly random in the adversary's view (and so is $\sk = X_1X_2$); a symmetric argument can be made for an adversarial server. In the man-in-the-middle setting, an adversary can attempt to generate a valid ciphertext after seeing another valid ciphertext from the honest user/server, so we need the encryption scheme to be non-malleable.\\

The one-time signature scheme effectively forces the adversary to pass $\msg{2}$ and $\msg{3}$ without modification as long as it passes $\msg{1}$. This is to prevent a man-in-the-middle adversary from gaining information by passing all ciphertexts but modifying the rest of the messages. Specifically, (if the signature scheme is removed) consider an adversary that passes $\msg{1} = A|B|C|D$, changes $\msg{2} = E|F|G|I|J$ to $E^{\frac{1}{2}}|F|G|I|J$, and changes $\msg{3} = K$ to $K^2$; this would cause $\sk_S = \sk_U^2$. In other words, the adversary (that does not know the password) causes the two parties' session keys to be unequal but correlated, which is not allowed by the security of PAKE. Furthermore, to prevent the adversary from plugging in its own verification key (and thus knowing the corresponding secret key), $\VK$ is included in the hash that produces $\alpha$. In this way if the adversary changes $\VK$ while keeping the ciphertext $A|B|C|D$, the ciphertext would become invalid.
	
	\subsection{Technical Overview}

In this section, we provide a high-level explanation of why the KOY protocol is insecure in the UC framework, and how the AGM circumvents the difficulty for the UC simulator.

	\paragraph{UC-insecurity of KOY.}
	Our attack relies on an adversary $\adv$ that completely disregards the presence of $\server$ and instead interacts with $\user$ while executing $\server$'s algorithm on its own. In particular, once the protocol is initiated by $\user$, $\adv$ assumes the role of the server (discarding the actual server in the process, which plays no part in the protocol) and receives $\msg{1} = \mathsf{VK}|A|B|C|D$. After this, $\adv$ runs the server's algorithm on $\user$'s password $\pw$ (i.e., we assume that $\adv$ makes a correct password guess) and computes $\msg{2} = E|F|G|I|J$. $\user$ then runs its session-key generating algorithm and outputs its session key $\sk = E^{r_1}F^{x_1}G^{y_1}(I/\pw)^{z_1}J^{w_1}$. At this point, $\adv$ (and $\env$) have all the information they need to run the server's session-key generating algorithm locally, which computes a session key equal to $\sk$ generated by $\user$.\\
	
	To see why a simulator $\simulator$ cannot simulate this adversary, we attempt an ideal-world execution and pinpoint where it fails. Since $\simulator$ is allowed to choose $\crs=(g_1, g_2, h, c, d)$, it can sample $g_1$ at random and set $h$ such that $h=g_1^{\ell}$ --- in other words, $\simulator$ chooses the ``CRS trapdoor'' $\ell = \log_{g_1} h$. After receiving the $\newsession$ command from $\pake$, $\simulator$ must simulate $\user$'s first message $\msg{1}$. Since $\simulator$ does not know the password, at this point it must (effectively) guess some $\pw'$ at random; that is, in $\msg{1}$, $C=h^{r_1}\cdot\pw'$ where $\pw'$ can be no better than a random password sampled from the dictionary. ($C$ is indistinguishable from the correct value due to the security of Cramer--Shoup encryption.) After $\env$ responds with $\msg{2} = E|F|G|I|J$, since $F = g_1^{r_2}$ and $I = h^{r_2} \cdot \pw$, $\simulator$ can extract $\pw$ as $I/F^\ell$. Once this has been done, $\simulator$ can send a $\testpwd$ command to $\pake$ on the correct $\pw$; $\pake$ would mark the $\user$ session $\compromised$ and thus allow $\simulator$ to choose $\user$'s session key $\sk$ (which has to be consistent with the session key $\user$ computes in the real world).\footnote{A $\testpwd$ \textit{must} be run, since we require that $\msg{1}$ and $\msg{2}$ together with the randomness of the $\user$ and $\adv$ together determine $\sk$; allowing the simulation to proceed without a $\testpwd$ would result in $\pake$ outputting a uniformly random key.} This is where the game-based security and UC-security of PAKE diverge: in game-based security, all security guarantees are considered lost (and the simulation of the game can stop) once the adversary guesses the correct password; whereas in UC-security the simulation has to continue. The problem here is that \emph{even knowing the correct password $\pw$, $\simulator$ still cannot determine what $\sk$ should be}.\\

Recall that $\sk$ is the product of
\[
X_1 = E^{r_1} = A^{x_2}B^{y_2}(C/\pw)^{z_2}D^{w_2}
\]
and
\[
X_2 = K^{r_2} = F^{x_1}G^{y_1}(I/\pw)^{z_1}J^{w_1}
\]
Computing $X_2$ is not a problem for $\simulator$, since $\msg{2} = E|F|G|I|J$ is provided to $\simulator$ directly from the environment; $\simulator$ chose $x_1, y_1, z_1, w_1$ on its own; and $\simulator$ has extracted $\pw$. However, $\simulator$ is not able to compute $X_1$. At first glance, computing $X_1 = E^{r_1}$ might appear feasible as $\simulator$ received $E$ as part of $\msg{2}$ and sampled $r_1$ before sending $\msg{1}$. However, the problem is that \emph{the password guess $\pw'$ that $\simulator$ uses while generating $\msg{1}$ is likely incorrect}; as a result, $E^{r_1}$ that $\simulator$ computes is actually equal to $A^{x_2}B^{y_2}(C/\pw')^{z_2}D^{w_2}$, whereas the correct value should be $A^{x_2}B^{y_2}(C/\pw)^{z_2}D^{w_2}$. This means that $\simulator$ must know $(\pw/\pw')^{z_2}$ in order to compute the correct $\sk$, which is infeasible unless $\pw'=\pw$ (whose probability is $1/\mathcal{|D|}$).

\paragraph{UC-AGM security of KOY.}
By examining the game-based security proof of the KOY protocol, one can see that the above attack is essentially the only scenario where the UC-security can be broken; in particular, the cases where the adversary assumes the role of the user, as well as the adversary makes an incorrect password guess (by sending the Cramer--Shoup ciphertext of some $\pw^* \neq \pw$, or sending some group elements that are not a Cramer--Shoup ciphertext), can be simulated in UC without any issue. However, the above attack can be simulated in the AGM, as the simulator $\simulator$ can extract $x_2, y_2, z_2, w_2$ from an algebraic environment.\\

In more detail, suppose $\env$ runs the above attack and sends $E$ as part of $\msg{2}$. At this point all group elements that $\env$ has seen are $g_1,g_2,h,c,d$ from $\crs$, $A,B,C,D$ from $\msg{1}$, and $\pw$. An algebraic $\env$ must ``explain'' how $E$ is computed; for now let's ignore $A,B,C,D,\pw$ and assume $\env$ computes
\[
E = g_1^{x_2}g_2^{y_2}h^{z_2}(cd^\alpha)^{w_2}
\]
In the real world $\user$ would compute $X_1 = E^{r_1}$, which is equal to $A^{x_2}B^{y_2}(C/\pw)^{z_2}D^{w_2}$. In the ideal world, as explained above, $\simulator$ cannot compute $X_1$ as $E^{r_1}$ since it chose the wrong $\pw'$ with high probability while generating $A,B,C,D$; however, after extracting the correct $\pw$ from $\msg{2}$, $\simulator$ can still compute $X_1$ as $A^{x_2}B^{y_2}(C/\pw)^{z_2}D^{w_2}$, since now it sees the algebraic coefficients $x_2,y_2,z_2,w_2$ from $\env$. In this way $\simulator$ generates $X_1$ that is indistinguishable from the real world.\\

Next, suppose $\env$ computes
\[
E = g_1^{x_2}g_2^{y_2}h^{z_2}(cd^\alpha)^{w_2}A^{x_2'}B^{y_2'}\left(\frac{C}{\pw}\right)^{z_2'}D^{w_2'}
\]
The argument in this case is more involved. In the real world $\user$ would compute
\begin{align*}
X_1 &= E^{r_1} \\
    &= g_1^{r_1x_2+r_1^2 x_2'} g_2^{r_1y_2+r_1^2 y_2'} h^{r_1z_2+r_1^2 z_2'} (cd^\alpha)^{r_1w_2+r_1^2 w_2'} \\
    &= A^{x_2+r_1x_2'} B^{y_2+r_1y_2'} \left(\frac{C}{\pw}\right)^{z_2+r_1z_2'} D^{w_2+r_1w_2'}
\end{align*}	
Similar to above, in the ideal world $\simulator$ can compute $X_1$ according to the last equation. However, now the expression of $X_1$ involves $r_1$, which poses a potential issue while arguing for indistinguishability. At some point in the proof, we need to replace the real Cramer--Shoup ciphertext with an encryption of some ``dummy'' $\pw'$; therefore, we need Cramer--Shoup to be secure \emph{even if the adversary additionally sees $X_1$} --- otherwise $\env$ can distinguish the real world (which uses the real ciphertext) and the ideal world (which uses the ``dummy''ciphertext) by looking at \emph{both} $\msg{1}$ and $\sk$. The problem is that unlike the simpler case above, here the reduction to the security of Cramer--Shoup cannot simulate $X_1$ trivially, since it does not know the randomness $r_1$.

\xjy{This is actually interesting, and I am not sure how the argument would go. Perhaps the starting point is to consider ElGamal first: roughly, you need ElGamal to be secure even if the ciphertext is something like $(g^r, h^r\cdot m, g^{r^2}, h^{r^2})$. My intuition is that you probably need some sort of non-standard assumption (similar to square Diffie--Hellman)? I think we can do ElGamal as a warm-up (perhaps even write it down) and then move on to Cramer--Shoup}
	\subsection{Proof of UC-Insecurity}

	\begin{theorem}
		Assuming the hardness of fixed-CDH, the protocol of [KOY] does not UC-realize $\mathcal{F}_\mathsf{pake}$ in the $\mathcal{F}_{\mathsf{crs}}$-hybrid model.
	\end{theorem}

	\begin{proof}
	
	Consider the environment $\env$ in \Cref{fig:adv} and the dummy adversary. It follows from the correctness of the protocol that in the real-world protocol execution $\env$ always outputs $1$, since the algorithm of $\env$ and $\adv$ is the same as that of an honest server. At a high level, we will show that any simulator that successfully simulates the protocol against $\env$ in the ideal world can be used to solve arbitrary instances of fixed-CDH.\\
	
	\begin{figure}[h]
		\begin{framed}
% [
%			linecolor=black,
%			linewidth=1pt,
%			roundcorner=5pt,
%			backgroundcolor=white,
%			userdefinedwidth=\textwidth,
%			]
			\vspace{2mm}
			\textbf{\underline{Environment $\env$:}}
			\begin{enumerate}\setcounter{enumi}{-1}
                \item \xjy{$\env$ receives the CRS $(g_1, g_2, h, c, d)$}.
				\item $\env$ selects $\pw\xleftarrow{\$}\mathcal{PW}$, where $\mathcal{PW}\subseteq\mathbb{G}$ is the password dictionary. It then sends $(\newsession,\mathsf{sid},\user,\server,\pw)$ to $\user$.
				\item $\env$ receives $\msg{1} = \mathsf{sid}|\mathsf{VK}|A|B|C|D$ from $\adv$ and samples $x_2, y_2, z_2, w_2, r_2\xleftarrow{\$}\mathbb{Z}_q$. It then sets 
				\begin{align*}
					\alpha'&:=H(A|B|C|D)\\
					E &:= g_1^{x_2}g_2^{y_2}h^{z_2}(cd^{\alpha'})^{w_2}\\
					F &:= g_1^{r_2}\\
					G &:= g_2^{r_2}\\
					I &:= h^{r_2}\cdot\pw\\
					\beta &:= H(\msg{1}|E|F|G|I)\\
					J &:= (cd^{\beta})^{w_2}
				\end{align*}
			and instructs $\adv$ to send $\msg{2} := \mathsf{sid}|E|F|G|I|J$ to $\user$. 
			\item $\env$ receives $\msg{3}=\mathsf{sid}|K|\sigma$ from $\adv$ and $(\mathsf{sid}, \sk)$ from $\user$.
			\item $\env$ sets $C':=C/\pw$ and then checks if $\mathsf{Vrfy}_{\mathsf{VK}}(\msg{1}|\msg{2}|K,\sigma)=1$. If yes, it computes $\sk_S:=A^{x_2}B^{y_2}(C')^{z_2}D^{w_2}K^{r_2}$ and outputs $1$ if $\sk_S=\sk$. If either of the two checks fails, it outputs $0$.
			\end{enumerate}
			\vspace{2mm}
		\end{framed}
		\caption{Our Setup.}
		\label{fig:adv}
	\end{figure}

	Assume that there exists a negligible function $\varepsilon:=\varepsilon(\lambda)$ such that there exists a ``successful'' simulator $\simulator$ for which $\env$ outputs $1$ with probability $1-\varepsilon$ in the ideal world. First, assume that CDH is hard over $(\mathbb{G}, p, h)$. Consider reduction $\mathcal{R}$ which does the following. \xjy{that runs the simulator $\simulator$ as follows (note that $\mathcal{R}$ plays the role of the environment $\env$, the PAKE functionality $\pake$, and the dummy parties $\user$ and $\server$ combined):}\\
	
	\nam{I'm not completely convinced if this is the best way to write this; the proof is correct in the essentials, but I think this needs to be rewritten since I'm not sure what formal part $\mathcal{R}$ is playing here...}
	
	\begin{enumerate}\setcounter{enumi}{-1}
        \item $\mathcal{R}$ receives $\mathsf{crs}=(g_1, g_2, h, c, d)$ from $\simulator$, outputs $h$ to its challenger, and receives $(h^a, h^b)$ where $a,b\rgets\mathbb{Z}_p$. \xjy{I moved ``receiving the CRS'' as the 0th step of the formal description of the reduction (and the environment)} \xjy{Is the group order $p$ or $q$? It is called $p$ in this step but $q$ in the second step}
		\item $\mathcal{R}$ sends $(\newsession,\mathsf{sid}, \user, \server)$ to $\simulator$.
		\item $\mathcal{R}$ waits to receive $\msg{1} = \mathsf{sid}|\mathsf{VK}|A|B|C|D$ in response (as the first message from $\user$ to $\server$). It then sets $\pw:=C/h^b$ and samples $x_2,y_2,w_2,r_2\rgets\mathbb{Z}_q$, setting
		\begin{align*}
			\alpha'&:=H(PIDs|A|B|C|D)\\
			E &:= g_1^{x_2}g_2^{y_2}h^{a}(cd^{\alpha'})^{w_2}\\
			F &:= g_1^{r_2}\\
			G &:= g_2^{r_2}\\
			I &:= h^{r_2}\cdot\pw\\
			\beta &:= H(\msg{1}|E|F|G|I)\\
			J &:= (cd^{\beta})^{w_2}
		\end{align*}
		i.e., the same computation as that of the honest server with a special choice of $\pw$ and $z_2$ \xjy{the special choice of $\pw = C/h^b$ and $z_2 = a$}, and sends $\msg{2}:=\mathsf{sid}|E|F|G|I|J$ to $\simulator$.
		\item $\mathcal{R}$ receives $\msg{3}=\mathsf{sid}|K|\sigma$ (as the second message from $\user$ to $\server$) and $(\mathsf{sid},\sk)$ from $\simulator$ (as $\user$'s output to $\env$), and checks if $\mathsf{Vrfy}_\mathsf{VK}(\msg{1}|\msg{2}|K,\sigma)=1$. If yes, \xjy{If not, $\mathcal{R}$ aborts. Otherwise} it calculates 
		$$h'=\frac{\sk}{A^{x_2}B^{y_2}D^{w_2}K^{r_2}}.$$
		\item $\mathcal{R}$ outputs $h'$.		
	\end{enumerate}

		We recall that $\sk_C=\sk_S$ with probability $1-\varepsilon$ by the assumption. Clearly in the protocol execution, we have, setting $C'=C/\pw=h^b$,
		$$\sk_S = A^{x_2}B^{y_2}(C')^{z_2}D^{w_2}K^{r_2}=A^{x_2}B^{y_2}(h^b)^{a}D^{w_2}K^{r_2}$$ which clearly gives $h' = h^{ab}$ with probability $1-\varepsilon$, as claimed, contradicting the hardness of fixed-CDH.

\xjy{Note that $\simulator$'s view while interacting with $\mathcal{R}$ is identical to $\simulator$'s view in the ideal world with environment $\env$ in \Cref{fig:adv}; the difference is that $\env$ samples $\pw$ and $z_2$ on its own, whereas $\mathcal{R}$ sets $\pw = C/h^b$ and $z_2 = a$ --- which cannot be detected by $\simulator$. Let $C' = C/\pw = h^b$ and
$$\sk_S = A^{x_2}B^{y_2}(C')^{z_2}D^{w_2}K^{r_2}=A^{x_2}B^{y_2}(h^b)^{a}D^{w_2}K^{r_2}$$
as what $\env$ would compute in its step 4; $\env$ outputs 1 if and only if $\sk_S = \sk$, so by our assumption on $\simulator$, in $\mathcal{R}$'s interaction with $\simulator$, $\sk_S = \sk$ with probability $1-\varepsilon$. But this gives $h' = h^{ab}$ with probability $1-\varepsilon$, i.e., $\mathcal{R}$ wins with probability $1-\varepsilon$, contradicting the hardness of fixed-CDH.}
		\end{proof}
	
	
	\textcolor{gray}{//everything below is deprecated comments.}\\

	
	First, we will assume the hardness of CDH over the group $(\mathbb{G},g,p)$. Let Let $g^a,g^b$ be two elements where $a,b\xleftarrow{\$}\mathbb{Z}_p$. 
	
	Formally, assume that there exists a simulator $\simulator$ such that $\env$ always outputs $1$ in the ideal world. \xjy{Formally we cannot really assume this; need to say ``such that $\env$ outputs $1$ with all but negligible probability in the ideal world''. I am not entirely sure for now, but we probably need to be more specific and say ``there is a negligible function $\epsilon$ such that $\env$ outputs $1$ with probability $1-\epsilon$ in the ideal world.''} We will use this simulator to compute $g^{ab}$. \xjy{We will construct a reduction $\mathcal{R}$ that uses $\simulator$ to solve the CDH problem in $(\mathbb{G},g,p)$. (I think it's better to explicitly mention a reduction.)} Our technique works as follows: first, we send \xjy{everywhere you say ``we do something'', change it to ``$\mathcal{R}$ does something''} $(\newsession,\mathsf{sid},\user,\server,\pw)$ to $\pake$ \xjy{conceptually I think $\mathcal{R}$ should play the role of $\pake$ (if you are not sure what I am talking about, chat with me in our meeting or on Slack)} and recieve $\msg{1} = \mathsf{sid}|\mathsf{VK}|A|B|C|D$ in response from $\simulator$. We set $\pw = C/g^b$ and sample $x_2,y_2,z_2,w_2,r_2\xleftarrow{\$}\mathbb{Z}_q$. 
	\begin{align*}
		\alpha'&:=H(PIDs|A|B|C|D)\\
		E &:= g_1^{x_2}g_2^{y_2}h^{z_2}(cd^{\alpha'})^{w_2}\\
		F &:= g_1^{r_2}\\
		G &:= g_2^{r_2}\\
		I &:= h^{r_2}\cdot\pw\\
		\beta &:= H(\msg{1}|Server|E|F|G|I)\\
		J &:= (cd^{\beta})^{w_2}
	\end{align*}
	i.e., the same computation as that of the honest server with a special choice of $\pw$. Forwarding this to $\pake$, we recieve $\msg{3}=K|\mathsf{Sig}$ and $(\mathsf{sid}, \sk)$ in return.

	
	
	
	\pagebreak
	\appendix
	
%	\bibliographystyle{alpha}
%	\bibliography{references}
	
\end{document}










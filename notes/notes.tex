\documentclass[10pt,a4paper]{article}
\usepackage[margin=1in, includeheadfoot]{geometry}

\usepackage[dvipsnames]{xcolor}
\usepackage{amsmath}
\usepackage{amssymb}
\usepackage{amsthm}
%\usepackage{makeidx}
\usepackage{thmtools}
\usepackage{graphicx}
\usepackage{hyperref}
\usepackage{tikz}
\usepackage{algpseudocode}
\usepackage{multirow}
\usepackage{framed}
\usepackage[capitalise]{cleveref} % allow for \cref command


\usetikzlibrary{positioning}

\usepackage{thmtools}
\usepackage[skip=6pt]{parskip}

\usepackage{fancyhdr}
\pagestyle{fancy}

\fancyhead[L]{\textbf{On the UC-(In)Security of PAKE Protocols in the Plain Model}}
\fancyhead[R]{\thepage}
\fancyfoot{}

\setlength\parindent{0pt}

\hypersetup{
	colorlinks=true,
	linkcolor=red,
	filecolor=red,  
	citecolor=red,    
	urlcolor=blue,
	pdftitle={On the UC-(In)Security of PAKE Protocols in the Plain Model},
	pdfpagemode=FullScreen,
}

% \usepackage[framemethod=tikz]{mdframed}

\declaretheorem[name=Theorem, numberwithin=section]{theorem}
\declaretheorem[name=Lemma, numberwithin=section]{lemma}
\declaretheorem[name=Inequality, numberwithin=section]{inequality}
\declaretheorem[name=Definition, numberwithin=section]{definition}

\usepackage[english]{babel}
\title{\textbf{On the UC-(In)Security of PAKE Protocols in the Plain Model}}
\author{Naman Kumar, Jiayu Xu}

%------- General Macros

\newcommand{\np}{\mathbf{NP}}
\newcommand{\p}{\mathbf{P}}
\newcommand{\ip}{\mathbf{IP}}
\newcommand{\pspace}{\mathbf{PSPACE}}
\newcommand{\am}{\mathbf{AM}}
\newcommand{\ma}{\mathbf{MA}}
\newcommand{\F}{\mathbb{F}}
\newcommand{\E}{\mathbb{E}}
\newcommand{\rel}{\mathcal{R}}
\newcommand{\lang}{\mathcal{L}}
\newcommand{\query}{\mathcal{Q}}
\newcommand{\prover}{\mathcal{P}}
\newcommand{\verifier}{\mathcal{V}}
\newcommand{\decision}{\mathcal{D}}
\newcommand{\simulator}{\mathcal{S}}
\newcommand{\sat}{\mathsf{SAT}}
\newcommand{\nc}{\mathbf{NC}}
\newcommand{\hvzk}{\mathbf{HVZK}}
\newcommand{\zk}{\mathbf{ZK}}
\newcommand{\pzk}{\mathbf{PZK}}
\newcommand{\czk}{\mathbf{CZK}}
\newcommand{\szk}{\mathbf{SZK}}
\newcommand{\pc}{\mathsf{pc}}
\newcommand{\vc}{\mathsf{vc}}
\newcommand{\vr}{\mathsf{vr}}
\newcommand{\poly}{\mathsf{poly}}
\newcommand{\add}{\mathsf{add}}
\newcommand{\mul}{\mathsf{mul}}
\newcommand{\view}{\mathsf{View}}
\newcommand{\trace}{\textsc{Trace}}
\newcommand{\bpp}{\mathbf{BPP}}
\newcommand{\hilbert}{\mathcal{H}}
\newcommand{\ket}[1]{\left|#1\right\rangle}
\newcommand{\bra}[1]{\left\langle#1\right|}
\newcommand{\braket}[2]{\langle#1|#2\rangle}
\newcommand{\puncture}{\mathsf{Puncture}}

%---------- Macros

\newcommand{\env}{\mathcal{Z}}
\newcommand{\adv}{\mathcal{A}}
\newcommand{\red}{\mathcal{R}}
\newcommand{\pake}{\mathcal{F}_{\mathsf{PAKE}}}
\newcommand{\user}{\mathsf{User}}
\newcommand{\sk}{\mathsf{sk}}
\newcommand{\pw}{\mathsf{pw}}
\newcommand{\VK}{\mathsf{VK}}
\newcommand{\SK}{\mathsf{SK}}
\newcommand{\PK}{\mathsf{PK}}
\newcommand{\crs}{\mathsf{crs}}
\newcommand{\newsession}{\mathsf{NewSession}}
\newcommand{\testpwd}{\mathsf{TestPwd}}
\newcommand{\compromised}{\mathsf{compromised}}
\newcommand{\server}{\mathsf{Server}}
\newcommand{\msg}[1]{\mathsf{msg}_{#1}}
\newcommand{\rgets}{\xleftarrow{\$}}
\newcommand{\G}{\mathbb{G}}
\newcommand{\ct}{\mathsf{ct}}
\newcommand{\st}{\mathsf{st}}
\newcommand{\sign}{\mathsf{Sign}}

\newcommand{\enc}{\mathsf{Enc}}
\newcommand{\keygen}{\mathsf{KeyGen}}
\newcommand{\PKE}{\mathsf{PKE}}
\newcommand{\dec}{\mathsf{dec}}
\newcommand{\negl}{\mathsf{negl}}
\newcommand{\aux}{\mathsf{aux}}
\newcommand{\hk}{\mathsf{hk}}
\newcommand{\lab}{\mathsf{label}}

\newcommand{\drcca}{\mathsf{AdvDRCCA}}
\newcommand{\advsdh}{\mathsf{AdvSDH}}
\newcommand{\advddh}{\mathsf{AdvDDH}}
\newcommand{\game}{\mathbf{G}}
\newcommand{\estep}[1]{\mathbf{E#1}}
\newcommand{\dstep}[1]{\mathbf{D#1}}
\newcommand{\reject}{\mathsf{reject}}
\newcommand{\Z}{\mathbb{Z}}

\newcommand{\randdh}{\mathsf{RandDH}}
\newcommand{\ddh}{\mathsf{DH}}

\newcommand{\randsdh}{\mathsf{RandSDH}}
\newcommand{\sdh}{\mathsf{SDH}}

\def\xjy#1{\textcolor{blue}{Jiayu: #1}}
\def\nam#1{\textcolor{red}{Naman: #1}}

\begin{document}
	\maketitle
	
	This document contains a proof that the PAKE protocol of Katz, Ostrovsky and Yung (2001) is not UC-secure.

\section{UC-(In)Security of the KOY Protocol}
\subsection{Protocol Description}
\xjy{Insert the description of KOY (revised to fit the UC formality) here.} \nam{I've made edits, clarified the scheme.}

We provide a high-level description of the protocol along with a sketch of the security proof below. Let $\G$ be an algebraic group in which DDH is believed to be hard and define $\crs := (g_1, g_2, h, c, d)$ to be uniformly sampled from $\G^5$; we set this to be the common random string. Note that $\crs$ can be interpreted to be a degenerate Cramer-Shoup public key with an unknown secret key.

\begin{itemize}
  \item $\user$ begins by generating a pair of keys $(\VK,\SK)$ for a one-time signature scheme and encrypts its password $\pw$ with label $\VK$ using the Cramer-Shoup public key embedded in the CRS. Let $H$ be a collision-resistant hash function and $r_1$ be the randomness used during encryption. The resulting ciphertext consists of four group elements $\ct_1 := (A,B,C,D) = (g_1^{r_1}, g_2^{r_1}, h^{r_1} \cdot\pw, (cd^\alpha)^{r_1})$ where $\alpha = H(\VK, A, B, C)$. $\user$ then sends $\msg{1} := (\VK, \ct_1)$ to $\server$. Note that $A$, $B$, $C' = C/\pw$, $D$ are all of the form $g^{r_1}$ where $g$ is some group element that $\server$ can compute, so they also serve as a message in a Diffie--Hellman-like protocol.
  
  \item Upon receiving $(\VK,A,B,C,D)$, the $\server$ samples its `Diffie--Hellman exponents' $(x_2, y_2, z_2, w_2)$, and computes $E = g_1^{x_2}g_2^{y_2}h^{z_2}(cd^\alpha)^{w_2}$.\footnote{Note that $\alpha = H(\VK|A|B|C)$, so $\server$ must wait for $\user$'s message before proceeding.} Furthermore, $\server$ encrypts $\pw$ with label $(\msg{1}, E)$ using Cramer--Shoup encryption as done by $\user$ in $\msg{1}$. Let $r_2$ be the randomness used in encryption. The resulting ciphertext is $\ct_2:=(F,G,I,J) = (g_1^{r_2}, g_2^{r_2}, h^{r_2} \cdot\pw, (cd^\beta)^{r_2})$. $\server$ sends $\msg{2} = (E,\ct_2)$ to $\user$.
  
  \item Symmetrically, $\user$ upon receiving $(E,\ct_2)$ samples its own `Diffie--Hellman exponents' $(x_1,y_1,z_1,w_1)$ and computes $K = g_1^{x_1}g_2^{y_1}h^{z_1}(cd^\beta)^{w_1}$. It signs the protocol transcript $\st = \sign_\SK(\msg{1}|\msg{2}|K)$, and sends $\msg{3} = (K, \st)$ to the server.
  
  \item $\server$ verifies the signature $\st$ and aborts if it is invalid. Otherwise the session key $\sk$ is defined as the product of
      \[
      X_1 = E^{r_1} = A^{x_2}B^{y_2}(C/\pw)^{z_2}D^{w_2}
      \]
      and
      \[
      X_2 = K^{r_2} = F^{x_1}G^{y_1}(I/\pw)^{z_1}J^{w_1}
      \]
      $\user$ can compute $X_1$ as $E^{r_1}$ and $X_2$ as $F^{x_1}G^{y_1}(I/\pw)^{z_1}J^{w_1}$, whereas $\server$ can compute $X_1$ as $K^{r_2}$ and $X_2$ as $F^{x_1}G^{y_1}(I/\pw)^{z_1}J^{w_1}$.
\end{itemize}

\paragraph{Security.}
To perform authentication, $\user$ and $\server$ need to (implicitly) prove to each other that they know $\pw$. This is achieved as follows. Note that $X_1 = E^{r_1} = A^{x_2}B^{y_2}(C/\pw)^{z_2}D^{w_2}$ has the following property: given $E$ (but not $x_2,y_2,z_2,w_2$), if $A|B|C|D$ is a valid encryption of $\pw$, then knowing the randomness $r_1$ is sufficient for computing $X_1$; otherwise $X_1$ is a uniformly random group element.\footnote{Using the terminology of SPHF: $(x_2,y_2,z_2,w_2)$ is the hash key and $E$ is the corresponding projection key; the function is defined as $\mathsf{Hash}_{(x_2,y_2,z_2,w_2)}(m) = A^{x_2}B^{y_2}(C/m)^{z_2}D^{w_2}$.} Therefore, an adversarial user that does not know $\pw$, is not able to come up with a valid $A|B|C|D$, so $X_1$ is uniformly random in the adversary's view (and so is $\sk = X_1X_2$); a symmetric argument can be made for an adversarial server. In the man-in-the-middle setting, an adversary can attempt to generate a valid ciphertext after seeing another valid ciphertext from the honest user/server, so we need the encryption scheme to be non-malleable.

The one-time signature scheme effectively forces the adversary to pass $\msg{2}$ and $\msg{3}$ without modification as long as it passes $\msg{1}$. This is to prevent a man-in-the-middle adversary from gaining information by passing all ciphertexts but modifying the rest of the messages. Specifically, (if the signature scheme is removed) consider an adversary that passes $\msg{1} = A|B|C|D$, changes $\msg{2} = E|F|G|I|J$ to $E^{\frac{1}{2}}|F|G|I|J$, and changes $\msg{3} = K$ to $K^2$; this would cause $\sk_S = \sk_U^2$. In other words, the adversary (that does not know the password) causes the two parties' session keys to be unequal but correlated, which is not allowed by the security of PAKE. Furthermore, to prevent the adversary from plugging in its own verification key (and thus knowing the corresponding secret key), $\VK$ is included in the hash that produces $\alpha$. In this way if the adversary changes $\VK$ while keeping the ciphertext $A|B|C|D$, the ciphertext would become invalid.
	
	\subsection{Technical Overview}

In this section, we provide a high-level explanation of why the KOY protocol is insecure in the UC framework, and how the AGM circumvents the difficulty for the UC simulator.

	\paragraph{UC-insecurity of KOY.}
	Our attack relies on an adversary $\adv$ that completely disregards the presence of $\server$ and instead interacts with $\user$ while executing $\server$'s algorithm on its own. In particular, once the protocol is initiated by $\user$, $\adv$ assumes the role of the server (discarding the actual server in the process, which plays no part in the protocol) and receives $\msg{1} = \mathsf{VK}|A|B|C|D$. After this, $\adv$ runs the server's algorithm on $\user$'s password $\pw$ (i.e., we assume that $\adv$ makes a correct password guess) and computes $\msg{2} = E|F|G|I|J$. $\user$ then runs its session-key generating algorithm and outputs its session key $\sk = E^{r_1}F^{x_1}G^{y_1}(I/\pw)^{z_1}J^{w_1}$. At this point, $\adv$ (and $\env$) have all the information they need to run the server's session-key generating algorithm locally, which computes a session key equal to $\sk$ generated by $\user$.
	
	To see why a simulator $\simulator$ cannot simulate this adversary, we attempt an ideal-world execution and pinpoint where it fails. Since $\simulator$ is allowed to choose $\crs=(g_1, g_2, h, c, d)$, it can sample $g_1$ at random and set $h$ such that $h=g_1^{\ell}$ --- in other words, $\simulator$ chooses the ``CRS trapdoor'' $\ell = \log_{g_1} h$. After receiving the $\newsession$ command from $\pake$, $\simulator$ must simulate $\user$'s first message $\msg{1}$. Since $\simulator$ does not know the password, at this point it must (effectively) guess some $\pw'$ at random; that is, in $\msg{1}$, $C=h^{r_1}\cdot\pw'$ where $\pw'$ can be no better than a random password sampled from the dictionary. ($C$ is indistinguishable from the correct value due to the security of Cramer--Shoup encryption.) After $\env$ responds with $\msg{2} = E|F|G|I|J$, since $F = g_1^{r_2}$ and $I = h^{r_2} \cdot \pw$, $\simulator$ can extract $\pw$ as $I/F^\ell$. Once this has been done, $\simulator$ can send a $\testpwd$ command to $\pake$ on the correct $\pw$; $\pake$ would mark the $\user$ session $\compromised$ and thus allow $\simulator$ to choose $\user$'s session key $\sk$ (which has to be consistent with the session key $\user$ computes in the real world).\footnote{A $\testpwd$ \textit{must} be run, since we require that $\msg{1}$ and $\msg{2}$ together with the randomness of the $\user$ and $\adv$ together determine $\sk$; allowing the simulation to proceed without a $\testpwd$ would result in $\pake$ outputting a uniformly random key.} This is where the game-based security and UC-security of PAKE diverge: in game-based security, all security guarantees are considered lost (and the simulation of the game can stop) once the adversary guesses the correct password; whereas in UC-security the simulation has to continue. The problem here is that \emph{even knowing the correct password $\pw$, $\simulator$ still cannot determine what $\sk$ should be}.

Recall that $\sk$ is the product of
\[
X_1 = E^{r_1} = A^{x_2}B^{y_2}(C/\pw)^{z_2}D^{w_2}
\]
and
\[
X_2 = K^{r_2} = F^{x_1}G^{y_1}(I/\pw)^{z_1}J^{w_1}
\]
Computing $X_2$ is not a problem for $\simulator$, since $\msg{2} = E|F|G|I|J$ is provided to $\simulator$ directly from the environment; $\simulator$ chose $x_1, y_1, z_1, w_1$ on its own; and $\simulator$ has extracted $\pw$. However, $\simulator$ is not able to compute $X_1$. At first glance, computing $X_1 = E^{r_1}$ might appear feasible as $\simulator$ received $E$ as part of $\msg{2}$ and sampled $r_1$ before sending $\msg{1}$. However, the problem is that \emph{the password guess $\pw'$ that $\simulator$ uses while generating $\msg{1}$ is likely incorrect}; as a result, $E^{r_1}$ that $\simulator$ computes is actually equal to $A^{x_2}B^{y_2}(C/\pw')^{z_2}D^{w_2}$, whereas the correct value should be $A^{x_2}B^{y_2}(C/\pw)^{z_2}D^{w_2}$. This means that $\simulator$ must know $(\pw/\pw')^{z_2}$ in order to compute the correct $\sk$, which is infeasible unless $\pw'=\pw$ (whose probability is $1/\mathcal{|D|}$).

\paragraph{Simulating the session key in AGM.}
% By examining the game-based security proof of the KOY protocol, one can see that the above attack is essentially the only scenario where the UC-security can be broken; in particular, the cases where the adversary assumes the role of the user, as well as the adversary makes an incorrect password guess (by sending the Cramer--Shoup ciphertext of some $\pw^* \neq \pw$, or sending some group elements that are not a Cramer--Shoup ciphertext), can be simulated in UC without any issue. 

Our next critical observation is that the session key in the above attack can be simulated if we resort to the AGM, as the simulator $\simulator$ can extract $x_2, y_2, z_2, w_2$ from an algebraic environment. In more detail, suppose $\env$ runs the above attack and sends $E$ as part of $\msg{2}$. At this point all group elements that $\env$ has seen are $g_1,g_2,h,c,d$ from $\crs$, $A,B,C,D$ from $\msg{1}$, and $\pw$. An algebraic $\env$ must ``explain'' how $E$ is computed; for now let's ignore $A,B,C,D,\pw$ and assume $\env$ computes
\[
E = g_1^{x_2}g_2^{y_2}h^{z_2}(cd^\alpha)^{w_2}
\]
In the real world $\user$ would compute $X_1 = E^{r_1}$, which is equal to $A^{x_2}B^{y_2}(C/\pw)^{z_2}D^{w_2}$. In the ideal world, as explained above, $\simulator$ cannot compute $X_1$ as $E^{r_1}$ since it chose the wrong $\pw'$ with high probability while generating $A,B,C,D$; however, after extracting the correct $\pw$ from $\msg{2}$, $\simulator$ can still compute $X_1$ as $A^{x_2}B^{y_2}(C/\pw)^{z_2}D^{w_2}$, since now it sees the algebraic coefficients $x_2,y_2,z_2,w_2$ from $\env$. In this way $\simulator$ generates $X_1$ that is indistinguishable from the real world.

In the general case, suppose $\env$ computes
\[
E = g_1^{x_2}g_2^{y_2}h^{z_2}c^{w_2}d^{v_2}A^{x_2'}B^{y_2'}C^{z_2'}D^{w_2'}\pw^{p_2}
\]
In the real world $\user$ would compute
\begin{align*}
X_1 &= E^{r_1} \\
    &= g_1^{r_1x_2} g_2^{r_1y_2} h^{r_1z_2} c^{r_1w_2} d^{r_1v_2} A^{r_1x_2'} B^{r_1y_2'} C^{r_1z_2'} D^{r_1w_2'} \pw^{r_1p_2} \\
    &= g_1^{r_1x_2} g_2^{r_1y_2} \left(\frac{C}{\pw}\right)^{z_2} c^{r_1w_2} d^{r_1v_2} A^{r_1x_2'} B^{r_1y_2'} C^{r_1z_2'} D^{r_1w_2'} \pw^{r_1p_2}
\end{align*}	
Again, in the ideal world $\simulator$ can compute $X_1$ according to the last equation; the key point is that given $A|B|C|D$, the only place where the ``wrong'' password $\pw'$ is used lies in $h$, so we only need to ``correct'' $h^{r_1}$ from $C/\pw'$ to $C/\pw$. A difference is that now the expression of $X_1$ involves the Cramer--Shoup randomness $r_1$, which is not a problem for $\simulator$ as $\simulator$ chose $r_1$ itself while sending $\msg{1}$. (Of course, we have $A = g_1^{r_1}$, so the $g_1^{r_1x_2} A^{r_1x_2'}$ part can be rewritten as $A^{x_2+r_1x_2'}$, and so on. But this simplification is not necessary.)

\paragraph{Further subtleties in UC.}
As we have just seen, one critical difference between game-based security and UC-security is that in UC indistinguishability between the real view and the simulated view must remain \emph{even when the environment sees the key of a successfully attacked session}, whereas in the game-based setting the adversary simply wins (and the simulator can ``give up'' on simulating the session key) once the attack on a session is successful. While the session key itself can be simulated in the AGM, this poses another potential issue that is more subtle: after seeing the session key, the environment might go back and check the validity of previous protocol messages using this information.

In more detail, assume again that the adversary interacts with $\user$ by running $\server$'s algorithm on $\user$'s password $\pw$. When $\user$'s session completes, the environment $\env$ sees $\user$'s session key $\sk_U = X_1X_2$ and the last message $\msg{3} = K|\sigma$. Since $\env$ can compute $X_2$ as $K^{r_2}$ (since $r_2$ is chosen by $\env$ itself), it can recover $X_1 = E^{r_1}$ (where $r_1$ is the randomness used in $\msg{1}$). Recall that in the ideal world the Cramer--Shoup ciphertext $A|B|C|D$ generated by the simulator $\simulator$ is an encryption of some $\pw'$ that is unlikely to be the ``correct'' $\pw$, so $\env$ must not be able to detect this fact even after seeing $E^{r_1}$. In other words, (very roughly) \emph{Cramer--Shoup must be secure even if the adversary sees $E^{r_1}$ for some $E$ of its choice}.

Below we analyze two simple attacks using this strategy:
\begin{enumerate}
  \item Say $\env$ chooses $E = A = g_1^{r_1}$; then $E^{r_1} = g_1^{r_1^2}$. If 2-DL is easy in the group\footnote{The 2-DL problem is: given $(g^x,g^{x^2})$ for $x \gets \mathbb{Z}_q$, compute $x$. We do not know the relative hardness between 2-DL and DDH, and it has been shown that 2-DL and CDH are separate in the AGM \cite{C:BauFucLos20}.}, then $\env$ can recover $r_1$ after $\user$'s session completes, and check if $C = h^{r_1} \cdot \pw$. In the real world this equation holds, whereas in the ideal world it does not if $\simulator$ chose the ``wrong'' $\pw' \neq \pw$ to encrypt while simulating $\msg{1}$. In fact, it seems that for the KOY protocol to be UC-AGM-secure, we need the \xjy{FILL IN} assumption. \xjy{I think the assumption should be Decisional Square Diffie--Hellman (DSDH), which says that given $g^x$, it is hard to distinguish $g^{x^2}$ from random. Note that this implies 2-DL (given $(g^x,Y)$ where $Y$ is either $g^{x^2}$ or random, the DSDH solver can feed $(g^x,Y)$ to the 2-DL solver and distinguish by observing whether the 2-DL solver wins or not)}
  \item Say $\env$ chooses $E = c$; then $E^{r_1} = c^{r_1}$. But Cramer--Shoup is obviously \emph{not} CCA-secure if the adversary additionally sees $c^{r_1}$. What saves us here is that $\env$ sees $E^{r_1}$ \emph{only at the end of $\user$'s session}. Indeed, the reduction $\red$ to the security of Cramer--Shoup roughly works as follows:
      \begin{enumerate}
        \item $\red$ embeds the challenge ciphertext as $\msg{1}$, $\user$'s first message intercepted by $\adv$;
        \item When $\adv$ sends $\msg{1}^*$ to $\server$ and $\msg{2}^*$ to $\user$, $\red$ needs to query the decryption oracle to extract the password guesses contained in these two messages;
        \item Finally, if the password guess in $\msg{2}^*$ is correct, $\red$ needs to simulate $\sk_U$. (This step is not needed in the game-based proof.)\footnote{We omit the remaining steps which are to simulate $\msg{3}$, and on $\msg{3}^*$ simulate $\server$'s session key $\sk_S$, as they are inconsequential to our main point.}
      \end{enumerate}
      $c^{r_1}$ is needed only in step (c), which happens after all decryption oracle queries have been made. Therefore, we only need Cramer--Shoup to remain CCA-secure if the adversary \emph{is restricted to making all decryption oracle queries before learning $c^{r_1}$}. We will show that Cramer--Shoup indeed satisfies this property in \xjy{FILL IN} \xjy{I think this should work, but it needs to be double checked}
\end{enumerate}
By inspecting the game-based security proof of the KOY protocol, one can see that the above attacks are essentially the only scenarios where the UC-security might be broken, and the security argument for all other cases (including the adversary sending a message that contains an incorrect password guess, or modifying the signature) is essentially identical to the game-based proof. \xjy{I hope so...}

	\subsection{Proof of UC-Insecurity}

	\begin{theorem}
		Assuming the hardness of fixed-CDH, the protocol of [KOY] does not UC-realize $\mathcal{F}_\mathsf{pake}$ in the $\mathcal{F}_{\mathsf{crs}}$-hybrid model.
	\end{theorem}

	\begin{proof}
	
	Consider the environment $\env$ in \Cref{fig:adv} and the dummy adversary. It follows from the correctness of the protocol that in the real-world protocol execution $\env$ always outputs $1$, since the algorithm of $\env$ and $\adv$ is the same as that of an honest server. At a high level, we will show that any simulator that successfully simulates the protocol against $\env$ in the ideal world can be used to solve arbitrary instances of fixed-CDH.
	
	\begin{figure}[h]
		\begin{framed}
% [
%			linecolor=black,
%			linewidth=1pt,
%			roundcorner=5pt,
%			backgroundcolor=white,
%			userdefinedwidth=\textwidth,
%			]
			\vspace{2mm}
			\textbf{\underline{Environment $\env$:}}
			\begin{enumerate}\setcounter{enumi}{-1}
                \item $\env$ receives the CRS $(g_1, g_2, h, c, d)$.
				\item $\env$ selects $\pw\xleftarrow{\$}\mathcal{PW}$, where $\mathcal{PW}\subseteq\mathbb{G}$ is the password dictionary. It then sends $(\newsession,\mathsf{sid},\user,\server,\pw)$ to $\user$.
				\item $\env$ receives $\msg{1} = \mathsf{sid}|\mathsf{VK}|A|B|C|D$ from $\adv$ and samples $x_2, y_2, z_2, w_2, r_2\xleftarrow{\$}\mathbb{Z}_q$. It then sets 
				\begin{align*}
					\alpha'&:=H(A|B|C|D)\\
					E &:= g_1^{x_2}g_2^{y_2}h^{z_2}(cd^{\alpha'})^{w_2}\\
					F &:= g_1^{r_2}\\
					G &:= g_2^{r_2}\\
					I &:= h^{r_2}\cdot\pw\\
					\beta &:= H(\msg{1}|E|F|G|I)\\
					J &:= (cd^{\beta})^{w_2}
				\end{align*}
			and instructs $\adv$ to send $\msg{2} := \mathsf{sid}|E|F|G|I|J$ to $\user$. 
			\item $\env$ receives $\msg{3}=\mathsf{sid}|K|\sigma$ from $\adv$ and $(\mathsf{sid}, \sk)$ from $\user$.
			\item $\env$ sets $C':=C/\pw$ and then checks if $\mathsf{Vrfy}_{\mathsf{VK}}(\msg{1}|\msg{2}|K,\sigma)=1$. If yes, it computes $\sk_S:=A^{x_2}B^{y_2}(C')^{z_2}D^{w_2}K^{r_2}$ and outputs $1$ if $\sk_S=\sk$. If either of the two checks fails, it outputs $0$.
			\end{enumerate}
			\vspace{2mm}
		\end{framed}
		\caption{Our Setup.}
		\label{fig:adv}
	\end{figure}

	Assume that there exists a negligible function $\varepsilon:=\varepsilon(\lambda)$ such that there exists a ``successful'' simulator $\simulator$ for which $\env$ outputs $1$ with probability $1-\varepsilon$ in the ideal world. First, assume that CDH is hard over $(\mathbb{G}, p, h)$. Consider reduction $\mathcal{R}$ that runs the simulator $\simulator$ as follows (note that $\mathcal{R}$ plays the role of the environment $\env$, the PAKE functionality $\pake$, and the dummy parties $\user$ and $\server$ combined):
	
	% \nam{I'm not completely convinced if this is the best way to write this; the proof is correct in the essentials, but I think this needs to be rewritten since I'm not sure what formal part $\mathcal{R}$ is playing here...}
	
	\begin{enumerate}\setcounter{enumi}{-1}
        \item $\mathcal{R}$ receives $\mathsf{crs}=(g_1, g_2, h, c, d)$ from $\simulator$, outputs $h$ to its challenger, and receives $(h^a, h^b)$ where $a,b\rgets\mathbb{Z}_p$. 
        
        %\xjy{I moved ``receiving the CRS'' as the 0th step of the formal description of the reduction (and the environment)} \xjy{Is the group order $p$ or $q$? It is called $p$ in this step but $q$ in the second step}
        
		\item $\mathcal{R}$ sends $(\newsession,\mathsf{sid}, \user, \server)$ to $\simulator$.
		\item $\mathcal{R}$ waits to receive $\msg{1} = \mathsf{sid}|\mathsf{VK}|A|B|C|D$ in response (as the first message from $\user$ to $\server$). It then sets $\pw:=C/h^b$ and samples $x_2,y_2,w_2,r_2\rgets\mathbb{Z}_q$, setting
		\begin{align*}
			\alpha'&:=H(PIDs|A|B|C|D)\\
			E &:= g_1^{x_2}g_2^{y_2}h^{a}(cd^{\alpha'})^{w_2}\\
			F &:= g_1^{r_2}\\
			G &:= g_2^{r_2}\\
			I &:= h^{r_2}\cdot\pw\\
			\beta &:= H(\msg{1}|E|F|G|I)\\
			J &:= (cd^{\beta})^{w_2}
		\end{align*}
		i.e., the same computation as that of the honest server with the special choice of $\pw = C/h^b$ and $z_2 = a$, and sends $\msg{2}:=\mathsf{sid}|E|F|G|I|J$ to $\simulator$.
		\item $\mathcal{R}$ receives $\msg{3}=\mathsf{sid}|K|\sigma$ (as the second message from $\user$ to $\server$) and $(\mathsf{sid},\sk)$ from $\simulator$ (as $\user$'s output to $\env$), and checks if $\mathsf{Vrfy}_\mathsf{VK}(\msg{1}|\msg{2}|K,\sigma)=1$. If not, $\mathcal{R}$ aborts. Otherwise it calculates 
		$$h'=\frac{\sk}{A^{x_2}B^{y_2}D^{w_2}K^{r_2}}.$$
		\item $\mathcal{R}$ outputs $h'$.		
	\end{enumerate}

		Note that $\simulator$'s view while interacting with $\mathcal{R}$ is identical to $\simulator$'s view in the ideal world with environment $\env$ in \Cref{fig:adv}; the difference is that $\env$ samples $\pw$ and $z_2$ on its own, whereas $\mathcal{R}$ sets $\pw = C/h^b$ and $z_2 = a$ --- which cannot be detected by $\simulator$. Let $C' = C/\pw = h^b$ and
		$$\sk_S = A^{x_2}B^{y_2}(C')^{z_2}D^{w_2}K^{r_2}=A^{x_2}B^{y_2}(h^b)^{a}D^{w_2}K^{r_2}$$
		as what $\env$ would compute in its step 4; $\env$ outputs 1 if and only if $\sk_S = \sk$, so by our assumption on $\simulator$, in $\mathcal{R}$'s interaction with $\simulator$, $\sk_S = \sk$ with probability $1-\varepsilon$. But this gives $h' = h^{ab}$ with probability $1-\varepsilon$, i.e., $\mathcal{R}$ wins with probability $1-\varepsilon$, contradicting the hardness of fixed-CDH.
\end{proof}

	
	
	\section{Preliminaries}
	
	\subsection{Assumptions}
	
	Let $\mathcal{G}$ be an algorithm that on input $1^\lambda$ uniformly samples from a sequence of group distributions $(S_\lambda)_{\lambda\in\Z\geq 0}$ of group distributions of the form $\Gamma[\hat{\mathbb{G}}, \mathbb{G}, g, q]\in[S_\lambda]$, where $\hat{\mathbb{G}}$ is a finite multiplicative abelian group, $\G$ is a prime-order subgroup, $g$ is a generator of $\G$ and $q$ is the order of $\G$. We assume that group operations in $\G$ can be done in (expected) polynomial time in $\G$, including exponentiation, selecting a random group element, and membership. 
	
	Note that a random group element can be chosen by selecting $x\rgets\Z_q$ and computing $g^x$. We define $\bar\G$ to be $\G\setminus\{1\}$; since $q$ is prime, $\bar\G$ is the set of generators of $\G$. If $x\rgets\bar\G$, then we obtain a random generator.
	
	\subsubsection{The DDH Assumption}
	
	Let $\mathcal{G}$ be as above. We define for each $\lambda\in\Z_{\geq 0}$ and for all $\Gamma[\hat\G, \G, g, q]\in[S_\lambda]$ the sets 
	\begin{equation*}
		\randdh_{\lambda,\Gamma} := \{(g, g^x, g^y, g^{z}) : x,y\rgets\Z_q^{*}, z\rgets\Z_q\}
	\end{equation*}
	and
	\begin{equation*}
		\ddh_{\lambda,\Gamma} := \{(g, g^x, g^y, g^{xy}) : x,y\rgets\Z_q^{*}\}.
	\end{equation*}
	Where $\ddh_{\lambda,\Gamma}$ is the set of Diffie-Hellman Triples. For any probabilistic polynomial-time distinguishing algorithm $\adv$ and for all $\lambda\in\Z_{\geq 0}$, we define the DDH advantage of $\adv$ against $\mathcal{G}$ at $\lambda$ given $\Gamma$ as
	\begin{equation*}
		\advddh_{\mathcal{G},\adv}(\lambda | \Gamma) = \left|\Pr[\adv(1^\lambda,\Gamma,\rho) = 1 : \rho\rgets\randdh_{\lambda,\Gamma}] - \Pr[\adv(1^\lambda,\Gamma,\rho) = 1 : \rho\rgets\ddh_{\lambda,\Gamma}]\right|
	\end{equation*}
	The Decisional Diffie-Hellman (DDH) assumption for $\mathcal{G}$ states that for every probabilistic, polynomial-time algorithm $\adv$, there exists some negligible function $\negl$ such that for all $\lambda\in\Z_{\geq 0}$, $$\advddh_{\mathcal{G},\adv}(\lambda | \Gamma)\leq\negl(\lambda)$$ where $\Gamma\rgets[S_\lambda]$.
	
	\subsubsection{The SDH Assumption}
	
	Let $\mathcal{G}$ be as above. We define for each $\lambda\in\Z_{\geq 0}$ and for all $\Gamma[\hat\G, \G, g, q]\in[S_\lambda]$ the sets 
	\begin{equation*}
		\randsdh_{\lambda,\Gamma} := \{(g, g^x, g^{y}) : x,y\rgets\Z_q^{*}\}
	\end{equation*}
	and
	\begin{equation*}
		\sdh_{\lambda,\Gamma} := \{(g, g^x, g^{x^2}) : x\rgets\Z_q^{*}\}.
	\end{equation*}
	Where $\sdh_{\lambda,\Gamma}$ is the set of Square Diffie-Hellman Triples. For any probabilistic polynomial-time distinguishing algorithm $\adv$ and for all $\lambda\in\Z_{\geq 0}$, we define the SDH advantage of $\adv$ against $\mathcal{G}$ at $\lambda$ given $\Gamma$ as
	\begin{equation*}
		\advsdh_{\mathcal{G},\adv}(\lambda | \Gamma) = \left|\Pr[\adv(1^\lambda,\Gamma,\rho) = 1 : \rho\rgets\randsdh_{\lambda,\Gamma}] - \Pr[\adv(1^\lambda,\Gamma,\rho) = 1 : \rho\rgets\sdh_{\lambda,\Gamma}]\right|
	\end{equation*}
	The Square Diffie-Hellman (DDH) assumption for $\mathcal{G}$ states that for every probabilistic, polynomial-time algorithm $\adv$, there exists some negligible function $\negl$ such that for all $\lambda\in\Z_{\geq 0}$, $$\advsdh_{\mathcal{G},\adv}(\lambda | \Gamma)\leq\negl(\lambda)$$ where $\Gamma\rgets[S_\lambda]$.
	
	
	
	\subsection{Cramer-Shoup Encryption Scheme}
	
	\begin{figure}[ht!]
		\begin{framed}
			% [
			%			linecolor=black,
			%			linewidth=1pt,
			%			roundcorner=5pt,
			%			backgroundcolor=white,
			%			userdefinedwidth=\textwidth,
			%			]
			
			\vspace{2mm}
			\textbf{Key Generation:} On input $1^\lambda$ from $\lambda\in\mathbb{Z}_{\geq 0}$, compute
			\begin{align*}
				&\Gamma[\hat{\mathbb{G}}, \mathbb{G}, g, q]\rgets\mathcal{G}(1^\lambda); \hk\rgets\mathsf{HF.Keyspace}_{\lambda,\Gamma}\\
				&w\rgets\mathbb{Z}_q^{*}; x_1,x_2,y_1,y_2,z_1,z_2\rgets\mathbb{Z}_q\\
				&g_1\gets g; g_2\gets g^{w}; c = g_1^{x_1}g_2^{x_2}; d\gets g_1^{y_1}g_2^{y_2}; h\gets g_1^{z_1}g_2^{z_2}
			\end{align*}
			and output the public key $$\PK = (\Gamma,\hk,g_1,g_2,c,d,h)$$ and the secret key $$\SK = (\Gamma, \hk, x_1,x_2,y_1,y_2,z_1,z_2).$$
			
			\vspace{4mm}
			
			\textbf{Encryption:} Given $1^\lambda$ for $\lambda\in\mathbb{Z}_{\geq 0}$, a public key $$\PK=(\Gamma,\hk,g_1,g_2,c,d,h)\in[\mathcal{G}_\lambda]\times[\mathsf{HF.Keyspace}_{\lambda,\Gamma}]\times\G^4,$$ a message $m\in G$ and a label $\lab$, compute\\
			
			\begin{itemize}
				\item[]\textbf{$\estep{1}$}: $r\rgets\mathbb{Z}_q$;
				\item[]\textbf{$\estep{2}$}: $u_1\rgets g_1^r$;
				\item[]\textbf{$\estep{3}$}: $u_2\rgets g_2^r$;
				\item[]\textbf{$\estep{4}$}: $e'\gets h^r$;
				\item[]\textbf{$\estep{5}$}: $e\gets e'\cdot m$;
				\item[]\textbf{$\estep{6}$}: $\alpha\gets\mathsf{HF}^{\lambda,\Gamma}_{\hk}(\lab, u_1, u_2, e)$;
				\item[]\textbf{$\estep{7}$}: $v\gets (cd^\alpha)^r$;\\
			\end{itemize}
		
			and output the ciphertext $\psi = (u_1, u_2, e, v)$.
			
			\vspace{4mm}
			
			\textbf{Decryption:} Given $1^\lambda$ for $\lambda\in\mathbb{Z}_{\geq 0}$, a secret key $$\SK=(\Gamma, \hk, x_1,x_2,y_1,y_2,z_1,z_2)\in[\mathcal{G}_\lambda]\times[\mathsf{HF.Keyspace}_{\lambda,\Gamma}]\times\mathbb{Z}_q^{6},$$
			along with a ciphertext $\psi$ and a label $\lab$, do the following.\\
			
			\begin{itemize}
				\item[]\textbf{$\dstep{1}$}: Parse $\psi$ as a $4$-tuple $(u_1,u_2,e,v)\in\G^4$, output $\reject$ and halt if $\psi$ is not of this form.
				\item[]\textbf{$\dstep{2}$}: Test if $u_1, u_2$ and $e$ belong to $\G$; output $\reject$ and halt if this is not the case.
				\item[]\textbf{$\dstep{3}$}: Compute $v'\gets\mathsf{HF}^{\lambda,\Gamma}_{\hk}(\lab,u_1,u_2,e)$.
				\item[]\textbf{$\dstep{4}$}: Test if $v = u_1^{x_1+y_1\alpha}u_2^{x_2+y_2\alpha};$ output $\reject$ and halt if this is not the case.
				\item[]\textbf{$\dstep{5}$}: Compute $e' = u_1^{z_1}u_2^{z_2}$.
				\item[]\textbf{$\dstep{6}$}: Compute $m\gets e\cdot {e'}^{-1}$ and output $m$.
			\end{itemize}
			
			
		\end{framed}
		\caption{The Cramer-Shoup Encryption Scheme with Labels.}
		\label{fig:cs03}
	\end{figure}
	
	

	\bibliographystyle{alpha}
	\bibliography{references}
	
	\newpage
	\appendix
	
	\section{Security of the Cramer-Shoup Encryption Scheme with Delayed Reveal of the Secret Key}
	
	In this section we prove the delayed-reveal-CCA security of the Cramer-Shoup Cryptosystem. We first start by formally defining the delayed-reveal-CCA game.
	
	\subsection{Security Against Delayed-Reveal Chosen Ciphertext Attack}
	
	\begin{figure}[ht!]
		\begin{framed}
			% [
			%			linecolor=black,
			%			linewidth=1pt,
			%			roundcorner=5pt,
			%			backgroundcolor=white,
			%			userdefinedwidth=\textwidth,
			%			]
			\vspace{2mm}
			\textbf{Stage 1:} The adversary queries a \textit{key generation oracle}. The key generation oracle computes $(\PK,\SK)\leftarrow\enc.\keygen$ and responds with $\PK$.
			
			\vspace{2mm}
			
			\textbf{Stage 2:} The adversary makes a sequence of calls to a decryption oracle. For each  decryption oracle query, the adversary submits a ciphertext $\psi$, and the decryption
			oracle responds with $\PKE.\dec(1^\lambda, \SK,\psi)$.
			
			\vspace{2mm}
			
			\textbf{Stage 3:} The adversary submits two messages $m_0, m_1\in\PKE.\mathsf{MsgSpace}_{\lambda,\PK}$ to an encryption oracle. We require that $|m_0| = |m_1|$.
			
			On input $(m_0, m_1)$, the encryption oracle computes $$\sigma\rgets\{0,1\};\psi^{*}\rgets\PKE.\enc(1^\lambda,\PKE,m_\sigma)$$ and responds with the target ciphertext $\sigma^{*}$.
			
			\vspace{2mm}
			
			\textbf{Stage 4:} The adversary continues to make calls to the decryption oracle, subject only to the restriction that a submitted ciphertext $\psi$ is not identical to $\psi^{*}$.
			
			\vspace{2mm}
			
			\textbf{Stage 5:} The adversary submits a $\mathsf{halt}$ statement to an \textit{auxilliary oracle}, which responds with some auxiliary information $\mathsf{aux}$. From this point on the adversary no longer has access to the decryption oracle, and is not able to make any further queries.
			
			\vspace{2mm}
			
			\textbf{Stage 6:} The adversary outputs $\hat{\sigma}\in\{0,1\}$.
			
		\end{framed}
		\caption{The Delayed-Reveal-CCA game.}
		\label{fig:dcca}
	\end{figure}

	We define the delayed-reveal-CCA advantage with auxiliary information $\aux$ of the adversary $\adv$ against $\PKE$ at $\lambda$, denoted as $\drcca(\lambda)$ to be $|\Pr[\sigma = \hat{\sigma}] - 1/2|$ in the above attack game in \cref{fig:dcca} with corresponding value of $\aux$. Informally, we say that a public-key cryptosystem is delayed-reveal-CCA secure with auxiliary information $\aux$ if the probability that $\sigma = \hat{\sigma}$ is negligible, ie. the adversary is able to correctly guess which message was encrypted with no more than negligible probability. This leads to the formal definition as below.
	
	\begin{definition}[Security Against Delayed-Reveal-Chosen Ciphertext Attack (DR-CCA)]
		Let $\PKE$ be any public-key encryption scheme. We say that $\PKE$ is delayed-reveal-CCA secure with auxiliary information $\aux$ if there exists some negligible function such that for all $\lambda\in\Z_{\geq 0}$, $$\drcca(\lambda)\leq\negl(\lambda).$$
	\end{definition}

	With this in hand, we can now formally state the theorem.

	\begin{theorem}
		\label{thm:drcca-cs}
		If the DDH and the SDH assumptions hold for $\mathbb{G}$ and the hash function $H$ is Target Collision-Resistant, then the Cramer-Shoup Encryption Scheme with Labels (\cref{fig:cs03}) is delayed-reveal-CCA secure with auxiliary information $\aux = (x_1,x_2,y_1,y_2,g_1^{r^2},g_2^{r^2})$.
	\end{theorem}

	\subsection{Analysis of DR-CCA Security of the Cramer-Shoup Encryption Scheme}

	\subsubsection{Overview}

	Our argument proceeds very closely to the proof of the CCA-security of the Cramer-Shoup encryption scheme outlined in Section 6.2 of \cite{cs01}. The argument follows a series of games starting at game $\game_0$, which corresponds to the actual attack, and ending at $\game_8$, which is a game in which the ciphertext returned to the adversary is completely independent of the hidden bit $\sigma$. In each game, $\sigma$ takes on identical values. We now define a few helpful variables to quantify the adversary's advantage in each game. Let $T_i$ be the event in game $\game_i$ that $\hat{\sigma}=\sigma$. We will show for each game that $|\Pr[T_{i-1} - \Pr[T_i]]$ is negligible. Game $\game_8$, which is completely independent of $\sigma$, naturally has probability $\Pr[T_8]=1/2$, because in this game the adversary can do no better than a coin toss. We also introduce the variable $\drcca_i(\lambda)$ to represent the \textit{advantage} of the adversary in game $\game_i$, which is defined as $|\Pr[T_i]-1/2|$. Ultimately our goal is to show that $\drcca_0(\lambda)\leq\negl(\lambda)$.
	
	The probability is taken over the following mutually independent random variables:	\begin{enumerate}
		\item The internal coin tosses of $\adv$.
		\item The values $\hk,w,x_1,x_2,y_1,y_2,z_1,z_2$ generated independently by the key generation algorithm.
		\item The values $\sigma\in\{0,1\}$ and $r\in\Z_q$ which are generated by the encryption oracle.
	\end{enumerate}
	
	Before continuing formally, we outline a brief sketch of each of the games.
	
	\begin{itemize}
		\item In game $\game_1$ we make some technical changes to the encryption algorithm which have no effect on the adversary's view. After game $\game_5$, the randomness $r$ will be largely meaningless and thus cannot be used for correct encryption. This game ensures that encryption can occur without using $r$.
		\item Game $\game_2$ moves part of the auxiliary information, $\aux_1 =(g_1^{r^2}, g_2^{r^2})$ as an output of the decryption oracle. This can only increase the power of the adversary, since it is also allowed to make decryption oracle queries that involve the auxiliary information.
		\item Game $\game_3$ replaces $g_i^{r^2}$ with random values, which follows from SDH. It shows that the auxiliary information $\aux_1$ has no effect on the adversary's advantage.
		\item Game $\game_4$ is a technical game that completes the argument of $\game_3$.
		\item Game $\game_5$ replaces $u_2$ with a random value, which follows from DDH. This breaks the correlation between $u_1, u_2$ and $h^r\cdot m_\sigma$, allowing $m_\sigma$ to be replaced by a completely random value in future games.
		\item Game $\game_6$ is really the core of the proof. In this game the decryption oracle is modified so that it additionally checks whether $u_2$ is indeed $g_2^r$ for some $r$. The indistinguishability of this game from the previous one follows from the fact that the adversary cannot with any nontrivial probability submit a valid ciphertext which does not carry the correlation $(g_1, g_2, h)$ that is given by the public key. In particular, note that $\game_5$ ensures that the target ciphertext $\psi^{*}$ does \textit{not} have that particular correlation. The immediate implication is that the adversary cannot with any nontrivial probability modify the target ciphertext $\psi^{*}$ in such a way that obtains even a different, valid ciphertext, let alone one that provides information about $m_\sigma$. However, the argument of indistinguishability is nontrivial and spreads across more games.
		\item In game $\game_7$, $m_\sigma$ is replaced by a uniformly random value. We will show that both the advantage of this game, as well as the probability that the adversary is able to come up with some `bad' ciphertext which could have potentially provided information about $m_\sigma$, is negligible.
		\item Finally, game $\game_8$ bounds the probability of a `bad' ciphertext by modifying the decryption oracle, which now rejects if the adversary is able to reuse the MAC part $v$ of the ciphertext. The indistinguishability argument shows that if the adversary could make this query, then it either (a) broke the hash function, or (b) made an extremely unlikely guess, which it simply does not have enough information to with any nontrivial probability.
	\end{itemize}

	This completes a description of the games. 
	
	\subsubsection{Notation} Before we proceed to the full proof, we describe some helpful notation. Most of this notation is borrowed from \cite{cs01}. Let $\adv$ be the DR-CCA adversary. We set the security parameter to be $\lambda\in\Z_{\geq 0}$ and the group description $\Gamma[\hat{\G},\G,g,q]\in[S_\lambda]$.
	
	Suppose that the public key is $(\Gamma,\hk,g_1, g_2, c,d,h)$ and that the secret key is $(\Gamma,\hk,x_1,x_2,y_1,y_2,z_1,z_2)$. Let $w:=\log_{g}g_2$ and define $x,y$ and $z$ as follows: $$x:=x_1+x_2w, y:= y_1+y_2w, z:= z_1+z_2w.$$ In particular, $x=\log_{g}c$, $y=\log_gd$ and $z=\log_g h$.
	
	When we deal with ciphertext $\psi$, we also define the following values:
	\begin{itemize}
		\item $u_1,u_2,e',e,v\in\G$, where $\psi = (u_1,u_2,e,v)$, with $h = u_1^{z_1}u_2^{z_2}$.
		\item $r,\hat{r} = \log_gu_2, \alpha, r_e = \log_ge, r_v = \log_g d$, and $t = x_1r + y_1r\alpha + x_2\hat{r}w + y_2\hat{r}\alpha w$.
	\end{itemize}

	We will also deal with the target ciphertext $\psi^{*}$. When dealing with this particular ciphertext, we denote each of the variables with a $*$, to obtain $u_1^{*}, u_2^{*}, {e'}^{*}, e^{*}, v^{*}, r^{*},\hat{r}^{*}, \alpha^{*}, r_e^{*}, r_v^{*}, t^{*}.$

	\subsubsection{Proof of \cref{thm:drcca-cs}}

	We now proceed to describe the games in detail. Game $\game_0$ is the original attack game.
	
	\textbf{Game $\game_1$.} This game is the same as $\game_1$ of \cite{cs01}. 
	
	We modify game $\game_0$ to obtain the new game, which is identical except for a small modification to the encryption oracle. Instead of using the encryption algorithm as given to compute the target ciphertext $\psi^{*}$, we use a modified encryption algorithm, in which the steps $\estep{4}$ and $\estep{7}$ are replaced by 
	\begin{itemize}
		\item[] $\estep{4}': e'\gets u_1^{z_1}u_2^{z_1}$;
		\item[] $\estep{7}': v\gets u_1^{x_1+y_1\alpha}u_2^{x_2+y_2\alpha}$.
	\end{itemize}
	These changes are purely conceptual and the values of $e'^{*}$ and $v^{*}$ are exactly the same in both games. It follows that $$\Pr[T_0]=\Pr[T_1].$$
	
	\textbf{Game $\game_2$.} This is our first new game. This game is identical to $\game_1$ with the following modification.
	
	We modify the \textit{encryption} oracle in Stage $3$ such that along with a ciphertext $\sigma^{*}$, it also outputs the \textit{auxiliary information} $\aux_1 = (g_1^{r^2}, g_2^{r^2}) = (u_1^{*r}, u_2^{*r})$. Furthermore, we also modify the \textit{auxiliary} oracle such that it no longer outputs the previous string, instead outputting only $\aux_2 = (x_1,x_2,y_1,y_2)$. 
	
	Our analysis of this game is simple. Note that by providing the adversary the values $(g_1^{r^2}, g_2^{r^2})$ \textit{before} it can no longer make decryption oracle queries, we can only increase the power (and hence the advantage) of the adversary. It follows that $\drcca_1(\lambda)\leq\drcca_2(\lambda)$. Rephrasing, we can see that $$|\Pr[T_1]-1/2|\leq|\Pr[T_2]-1/2|$$ and hence it is enough to show that $|\Pr[T_2]-1/2|\leq\negl(\lambda)$. The assertion will follow immediately.\\
	
	\textbf{Game $\game_3$.} In this game we again modify the encryption oracle. Instead of outputting the ciphertext $(\psi^{*}, \aux_1) = ((u_1^{*}, u_2^{*}, e^{*}, v^{*}), (g_1^{r^2}, g_2^{r^2}))$, the oracle samples a uniform $s$ from $\mathbb{Z}_{q}$ and outputs $(\psi^{*}, \aux_1) = ((u_1^{*}, u_2^{*}, e^{*}, v^{*}), (g_1^{s}, g_2^{r^2}))$. 
	
	Note that the only difference between the games $\game_2$ and $\game_3$ is that the tuple $(g_1, u_1, \aux_1[0])$ is a uniformly distributed tuple from $\sdh_{\lambda,\Gamma}$ in $\game_2$ while it is a uniformly distributed tuple from $\randsdh_{\lambda,\Gamma}$ in $\game_3$. It is thus immediately clear that the indistinguishability of the two games follows from the hardness of SDH. More specifically, we show the following.
	
	\begin{lemma}
		\label{lem:sdh-game}
		There exists some polynomial-time probabilistic algorithm $\adv_\mathsf{SDH}$ such that $$|\Pr[T_3]-\Pr[T_2]|\leq\mathsf{AdvSDH}_{\adv_\mathsf{SDH}, \mathcal{G}}(\lambda|\Gamma).$$
	\end{lemma}
	\begin{proof}
		We will describe $\adv_\mathsf{SDH}$ in detail. The algorithm takes in as input $1^\lambda$, a group description $\Gamma[\hat{\G},\G,g,q]\in[S_\lambda]$ and some tuple $(g^a, g^{b})$. The algorithm interacts with the DR-CCA adversary $\adv$. First, it begins by computing
		$$\hk\rgets\mathsf{HF.Keyspace}_{\lambda, \Gamma}; w\rgets\Z^{*}_q; x_1,x_2,y_1,y_2,z_1,z_2\rgets\Z_q; c\gets g^{x_1+wx_2}; d\gets g^{y_1+wy_2}; h\gets g^{z_1+wz_2}.$$
		Using the above, it generates a public key $\PK = (\Gamma, \hk, g,g^w, c, d, h)$ and a secret key $\SK = (\Gamma,\hk,x_1,x_2,y_1,y_2,z_1,z_2)$, and passes $\PK$ to $\adv$. We now show the behaviour of $\adv_\mathsf{SDH}$ on encryption and decryption queries. Whenever it receives a ciphertext of the form $\psi = (u_1,u_2,e,v)$ to the decryption oracle, it uses $\SK$ to decrypt and outputs either $\bot$ or some $m$. Whenever $\adv$ submits $(m_0,m_1)$ to the encryption oracle, $\adv_\mathsf{SDH}$ samples a uniform $\sigma\rgets\{0,1\}$ and computes
		$$u_1^{*} = g^a; u_2^{*} = (g^a)^w; e^{*} = {u_1^{*}}^{z_1}{u_2^{*}}^{z_2}\cdot m_\sigma; v^{*} =\mathsf{HF}_{\hk}^{\lambda,\Gamma}(u_1^{*},u_2^{*},e^{*}); v^{*} = {u_1^{*}}^{x_1+v^{*}y_1}{u_2^{*}}^{x_2+v^{*}y_2}$$ and outputs the ciphertext--auxiliary information pair $(\psi^{*},\aux_1) = ((u_1^{*},u_2^{*},e^{*},v^{*}), (g^{b}, (g^{b})^w).$ On input $\mathsf{halt}$ to the auxiliary oracle, $\adv_\mathsf{SDH}$ outputs $(x_1,x_2,y_1,y_2)$.
		
		Once the game in $\game_2$ has been perfectly simulated, the adversary outputs $1$ if $\hat{\sigma} = \sigma$ and $0$ otherwise. It is clear from the above simulation that for any fixed $\lambda$ and $\Gamma$, 
		\begin{align*}
			\Pr[T_2] &= \Pr[\adv_\mathsf{SDH}(1^\lambda,\Gamma,\rho) = 1 : \rho\rgets\sdh_{\lambda,\Gamma}]\\
			\Pr[T_3] &= \Pr[\adv_\mathsf{SDH}(1^\lambda,\Gamma,\rho) = 1 : \rho\rgets\randsdh_{\lambda,\Gamma}].
		\end{align*}
		It immediately follows that 
		$$|\Pr[T_3]-\Pr[T_2]|\leq\mathsf{AdvSDH}_{\adv_\mathsf{SDH}, \mathcal{G}}(\lambda|\Gamma).$$
	\end{proof}
	
	\textbf{Game $\game_4$.} We modify the encryption algorithm again. Instead of outputting the ciphertext $(\psi^{*}, \aux_1) = ((u_1^{*}, u_2^{*}, e^{*}, v^{*}), (g_1^{s}, g_2^{r^2}))$ as before, the oracle samples a uniform $s'$ from $\mathbb{Z}_{q}$ and outputs $(\psi^{*}, \aux_1) = ((u_1^{*}, u_2^{*}, e^{*}, v^{*}), (g_1^{s}, g_2^{s'}))$. 
	
	The proof essentially follows the same as that of the previous game, and can be omitted.
	
	\textbf{Game $\game_5$.} This is game $\game_2$ of the proof of \cite{cs01}. We modify the encryption oracle, replacing step $\estep{3}$ of the encryption algorithm with
	\begin{itemize}
		\item[] $\estep{3}':$ $\hat{r}\rgets\Z_q\setminus\{r\}; u_2\gets g_2^{\hat{r}}$.
	\end{itemize}
	In the games up to $\game_4$ we had $r^{*} = \hat{r}^{*}$, however in game $\game_5$ $r^{*}$ and $\hat{r}^{*}$ are completely independent except that they cannot be equal. Note, however, that like in the previous game, we have a tuple $(g_1, g_2, u_1^{*}, u_2^{*})$ which in game $\game_4$ is uniformly sampled from $\ddh_{\lambda,\Gamma}$ while in game $\game_5$ is uniformly sampled from $\randdh_{\lambda,\Gamma}$. This leads to the following lemma, which says that any distinguisher for the two games immediately gives a statistical distinguisher for the DDH instance.
	
	\begin{lemma}
		There exists some probabilistic-polynomial time algorithm $\adv_\mathsf{DDH}$ such that $$|\Pr[T_5]-\Pr[T_4]|\leq\advddh_{\adv_\mathsf{DDH},\mathcal{G}}(\lambda | \Gamma).$$
	\end{lemma}
	\begin{proof}
		The proof of this lemma is borrowed from \cite{cs01} and follows the same broad argument as that of \cref{lem:sdh-game}. The adversary $\adv_\mathsf{DDH}$ is initiated with $1^\lambda$, a group description $\Gamma$, and a tuple $(g_2, u_1, u_2)\in\G^3$.
		
		The adversary $\adv$ is used in the same way, with the decryption and auxiliary oracles having the same description. The only difference is how the encryption oracle responds with the target ciphertext. In this case, when $\adv_\mathsf{DDH}$ receives $(m_0,m_1)$, it computes $\sigma\rgets\{0,1\}$ and the following values:
		$$e^{*} = {u_1^{*}}^{z_1}{u_2^{*}}^{z_2}\cdot m_\sigma; v^{*} =\mathsf{HF}_{\hk}^{\lambda,\Gamma}(u_1^{*},u_2^{*},e^{*}); v^{*} = {u_1^{*}}^{x_1+v^{*}y_1}{u_2^{*}}^{x_2+v^{*}y_2}.$$
		The auxiliary information $\aux_1$ is sampled in the same manner, with $g^s$ and $g^{s'}$ uniformly random elements of $\G$.
		
		Once the game in $\game_4$ has been perfectly simulated, the adversary outputs $1$ if $\hat{\sigma} = \sigma$ and $0$ otherwise. It is clear from the above simulation that for any fixed $\lambda$ and $\Gamma$, 
		\begin{align*}
			\Pr[T_4] &= \Pr[\adv_\mathsf{DDH}(1^\lambda,\Gamma,\rho) = 1 : \rho\rgets\ddh_{\lambda,\Gamma}]\\
			\Pr[T_5] &= \Pr[\adv_\mathsf{DDH}(1^\lambda,\Gamma,\rho) = 1 : \rho\rgets\randdh_{\lambda,\Gamma}].
		\end{align*}
		It immediately follows that 
		$$|\Pr[T_5]-\Pr[T_4]|\leq\mathsf{AdvDDH}_{\adv_\mathsf{DDH}, \mathcal{G}}(\lambda|\Gamma).$$
	\end{proof}

	\textbf{Game $\game_6$.} This is game $\game_3$ of the proof of \cite{cs01}. In this game, we modify the decryption oracle. Instead of using the original decryption oracle, we modify the oracle, replacing the steps $\dstep{4}$ and $\dstep{5}$ with:
	\begin{itemize}
		\item[]$\dstep{4}'$: Test if $u_2 = u_1^w$ and $v = u_1^{x+y\alpha}$. Output $\mathsf{reject}$ and halt if this is not the case.
		\item[]$\dstep{5}'$: $h\gets u_1^z$.
	\end{itemize}
	We first notice that the decryption oracle does not make any use of $x_i, y_i, z_i$ except indirectly. It is easy to see that the decryption oracle of game $\game_5$ correctly answers all queries that $\game_6$ does. However, it is possible that $\game_6$ may refuse to answer certain queries. Let $R_6$ be the event that there is some query which is asked to $\game_6$ which would have been answered correctly by $\game_5$, but is rejected by $\game_6$.
	
	We recall the following, which is Lemma 4 in \cite{cs01}.
	
	\begin{lemma}
		\label{lem:neg-wedge}
		Let $U_1$, $U_2$ and $F$ be events defined on some probability space. Suppose that the event $U_1\wedge\neg F$ occurs iff $U_2\wedge\neg F$ occurs. Then $|\Pr[U_1]-\Pr[U_2]|\leq\Pr[F]$.
	\end{lemma}

	We will apply this lemma in our analysis. Note that the event $T_5\wedge\neg R_6$ and $T_6\wedge\neg R_6$ are identical. By the lemma, we get $$|\Pr[T_3]-\Pr[T_2]|\leq\Pr[R_3]$$ and it suffices to bound $\Pr[R_3]$. This will be done in games $\game_7$ and $\game_8$.
	
	\textbf{Game $\game_7$.} This game is identical to $\game_6$, however in this game the message is randomly selected instead of being $m_\sigma$. In particular, we modify the encryption oracle and replace $\estep{5}$ with
	\begin{itemize}
		\item[]$\estep{5}'$: $r\rgets\Z_q$; $e\gets g^r$.
	\end{itemize}
	It follows that $\Pr[T_7] = 1/2$, because now $\psi^{*}$ does not even involve $\sigma$, so the adversary's output is completely independent of it. We define the event $R_7$ which is the same as $R_6$, ie. some ciphertext $\psi$ is submitted to the decryption oracle which is rejected by $\dstep{4}'$ but would have passed $\dstep{4}$.
	
	\begin{lemma} We have
		\label{lem:t6t7}
		\begin{align*}
			\Pr[T_6] &= \Pr[T_7]\\
			\Pr[R_6] &= \Pr[R_7]
		\end{align*}
	\end{lemma}

	Before we show the proof, we recall Lemma 9 from \cite{cs01}.
	
	\begin{lemma}
		\label{lem:independence}
		Let $k,n$ be integers with $1\leq k\leq n$ and let $K$ be a finite field. Consider a probability space with random variables $\vec{\alpha}\in K^{n\times 1}, \vec{\beta}=(\beta_1,\dots,\beta_k)^\top\in K^{k\times 1},\vec{\gamma}\in K^{k\times 1}$, and $M\in K^{k\times n}$, such that $\vec{\alpha}$ is uniformly distributed over $K^{n\times 1}$, $\vec{\beta}=M\vec{\alpha}+\vec{\gamma}$, and for $1\leq i\leq k$, the $i$th rows of $M$ and $\gamma$ are determined by $\beta_1,\dots,\beta_{i-1}$. 
		
		Then conditioning on any fixed values of $\beta_1,\dots,\beta_{k-1}$ such that the resulting matrix $M$ has rank $k$, the value of $\beta_k$ is uniformly distributed over $K$ in the resulting probability space.
	\end{lemma}

	We will require a less general version of this lemma, but it suffices for the proof below.

	\begin{proof}[Proof (of \cref{lem:t6t7})]
		Our proof is essentially the same as that of \cite{cs01}, however the presence of the auxiliary information requires a more subtle analysis.
		
		Consider the variable $X=(\mathsf{coins},\hk,w,x_1,x_2,y_1,y_2,\sigma,r^{*},\hat{r}^{*})$, where $\mathsf{coins}$ is the internal randomness of the PPT adversary $\adv$, and the quantity $z$. Note that $X$ has the same values in $\game_6$ and $\game_7$.
		
		Consider also the quantity $r^{*}$, which takes different values in $\game_6$ and $\game_7$. We can call these values $r_6^{*}$ and $r^{*}_7$ respectively. It is clear that both $R_6$ and $T_6$ are functions of $X,z$ and $r^{*}_6$. Also, the events $R_7$ and $T_7$ have the same functional dependence on $X,z$ and $r^{*}_7$. Thus, showing that the distributions
		$$(X, z, r^{*}_6)\cong(X, z, r^{*}_7)$$ is enough to show the lemma. Now observe that if $X$ and $z$ are fixed, $r^{*}_7$ is uniform over $\Z_q$. Furthermore, note that the uniformity of $r^{*}_7$ is \textit{completely} independent of $\aux = (x_1,x_2,y_1,y_2)$, since $e$ is completely independent of $\aux$ as well. Hence, knowledge of $\aux$ has no bearing on the independence of $r^{*}_7$, and it remains so even if $\aux$ is public.
		
		We show that the same is true for $r^{*}_3$. Consider
		\begin{equation*}
			\begin{pmatrix}
				z\\
				r^{*}_3
			\end{pmatrix}=
		\begin{pmatrix}
			1 & w\\
			r^{*} & w\hat{r}^{*}
		\end{pmatrix}\cdot
		\begin{pmatrix}
			z_1\\
			z_2
		\end{pmatrix}+
		\begin{pmatrix}
			0\\
			\log_gm_{\sigma}
		\end{pmatrix}.
		\end{equation*}
		Let the $2\times 2$ matrix be referred to as $M$. Then $M$ is fixed conditioned on $X$, but $z_1$ and $z_2$ are independently distributed over $\Z_q$ (even in the knowledge of $\aux$). Furthermore, $\det(M) = w(\hat{r}^{*} - r^{*})\neq 0$. The proposition then follows from \cref{lem:independence}.
	\end{proof}

	\textbf{Game $\game_8$.} This game is the same as $\game_7$, with a small modification to the decryption oracle. We modify it so that it applies a special rejection rule: if the adversary submits a ciphertext $\psi$ for decryption at a point \textit{after} the encryption oracle has been invoked, such that $(u_1, u_2, e)\neq (u_1^{*}, u_2^{*}, e^{*})$, but $v = v^{*}$, then the decryption oracle outputs $\mathsf{reject}$ and halts even before executing $\dstep{4}'$.
	
	We define two events:
	\begin{enumerate}
		\item $C_8$, which is the event that the adversary submits a ciphertext which is rejected by the special rule.
		\item $R_8$, which is the even that some ciphertext $\psi$ is submitted which is is passed by the special rule, however it is rejected by $\dstep{4}'$, and would have been accepted by $\dstep{4}$.
	\end{enumerate}

	Note that the games proceed identically till $C_8$ occurs. Particularly, $R_7\wedge\neg C_8$ and $R_8\wedge\neg C_8$ are identical. We have, using lemma \cref{lem:neg-wedge} that
	$$|\Pr[R_5]-\Pr[R_4]|\leq\Pr[C_5].$$
	
	We will show two lemmas that complete the proof. The first lemma shows that the special rejection rule can be broken if the underlying hash is broken, while the second shows that breaking the rule without breaking the hash requires an information-theoretically unlikely guess.
	
	\begin{lemma}
		\label{lem:hash}
		There is some probabilistic polynomial-time algorithm $\adv_\mathsf{HASH}$ such that $$\Pr[C_8]\leq\mathsf{AdvTCR}_{\mathsf{HF},\adv_\mathsf{HASH}}(\lambda|\Gamma).$$
	\end{lemma}

	\begin{lemma}
		\label{lem:bounding-r8}
		We have $$\Pr[R_8]\leq Q_\adv(\lambda)/q.$$
	\end{lemma}

	The proof of \cref{lem:hash} is identical to the proof of Lemma 7 presented in \cite{cs01}, hence we will not present it. The proof of \cref{lem:bounding-r8} is also similar, however must be adapted to our setting. 
	
	Together, we show that $|\Pr[R_8]-\Pr[R_6]|\leq\negl\implies|\Pr[T_6]-1/2|\leq\negl$, which proves the assertion that $|\Pr[T_0]-1/2|\leq\negl$. We complete the proof with of \cref{lem:bounding-r8} below.
	
	\begin{proof}[Proof (of \cref{lem:bounding-r8}).]
		Consider the proof of Lemma $8$ presented in \cite{cs01}. Both sub-parts of the proof rely on the fact that the vector $\aux = (x_1,x_2,y_1,y_2)$ is independent and uniformly distributed in the eyes of the adversary. Clearly this is not the case if $\aux$ is provided as auxiliary information. However, the proof only relies on the fact that $\aux$ be uniform and independent \textit{at the time the decryption query is made}. Indeed, by the definition of the DR-CCA game, it is clear that once the auxiliary oracle is queried, no more decryption queries are allowed to be made. Hence, at the time any decryption query is made, $\aux$ is uniform and independent, and the proof of Lemma $8$ follows without any additional concerns.
	\end{proof}
	
	With this, we have shown that the Cramer-Shoup encryption scheme as defined in \cref{fig:cs03} is DR-CCA secure with auxiliary information $\aux = (x_1,x_2,y_1,y_2, g_1^{r^2}, g_2^{r^2})$.
	
	
	
	
	
	
	
	
	
	
	
	
	
	
	
\end{document}